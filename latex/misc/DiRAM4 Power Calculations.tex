% Sample file: math.tex
\documentclass{article}
\usepackage{mathtools}% http://ctan.org/pkg/mathtools
%\usepackage{amsmath}
\usepackage{geometry} 
\usepackage{units}
\usepackage[binary-units=true]{siunitx}
\usepackage{caption}

\DeclarePairedDelimiter\ceil{\lceil}{\rceil}
\DeclarePairedDelimiter\floor{\lfloor}{\rfloor}

\geometry{paperwidth=3in,paperheight=2.00in}  
\geometry{left=0.1in,right=0.1in}  
\geometry{top=0.1in,bottom=0.1in}  

 \begin{document}
 \begin{flushleft}


 \eject \pdfpagewidth=6in \pdfpageheight=1.75in
 
 \begin{table}[ht]
\caption*{DiRAM4 Energy}
\centering
\begin{tabular}{c c }
%\begin{tabular}{| c | c | c | c | c | c |}
\hline\hline
Operation & Energy (pJ) \\
\hline
Page Open & 100 \\
Page Close &  320 \\
Page Read/Write & 64 \\
NOP & 20 \\
Page Refresh & 320 \\ [1ex]
 \hline
\end{tabular}
\label{table:DiRAM4 Energy}
\end{table}

 \newpage
 \eject \pdfpagewidth=3in \pdfpageheight=2in
 
\begin{table}[ht]
\caption*{DiRAM4 Read Sequence}
\centering
\begin{tabular}{c c c c c c c c}
%\begin{tabular}{| c | c | c | c | c | c |}
\hline\hline
Operation & Number \\
\hline
Page Open & 32 \\
Page Close & 32 \\
Page Read & 32 \\
NOP & 32 \\ [1ex]
 \hline
\end{tabular}
\label{table:DRAM Read Sequence}
\end{table}

 \newpage

\eject \pdfpagewidth=6.5in \pdfpageheight=1.5in
\begin{alignat}{2}
\label{eq:Energy per Cycle}
\text{Average energy per cycle per port} (pJ) &= \frac{\text{\#PO's} \cdot E_{po}+\text{\#PC's} \cdot E_{pc}+\text{\#Read's} \cdot E_{cr}+\text{\#NOP's} \cdot E_{no}}{\text{Total number of cycles}}  \nonumber \\
 &= \frac{32 \cdot 100+32 \cdot 320+32 \cdot 64+32 \cdot 20}{128}  \nonumber \\
 &=\SI{126}{\pico\joule} \nonumber
\end{alignat}

\newpage
\eject \pdfpagewidth=3in \pdfpageheight=1.5in
\begin{alignat}{2}
%\label{eq:Maximum Power}
\text{Maximum Power} &= \frac{energy}{cycle} \cdot \frac{cycles}{sec} \cdot ports \nonumber \\
 &=\num{126d-12} \cdot \num{2d9} \cdot 64 \nonumber \\
 &= \SI{16.12}{\watt} \nonumber
\end{alignat}

\newpage
\eject \pdfpagewidth=3in \pdfpageheight=1in
\begin{alignat}{2}
%\label{eq:Maximum Power}
\text{Actual Power} &\approx \frac{\text{54Tbps}}{\text{131Tbps}} \cdot 16.12 =  \SI{6.65}{\watt}\nonumber
\end{alignat}

\newpage



 \end{flushleft}
 \end{document}
  
 