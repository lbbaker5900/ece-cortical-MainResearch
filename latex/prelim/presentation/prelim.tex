% Sample file: math.tex
\documentclass{article}
\usepackage{mathtools}% http://ctan.org/pkg/mathtools
\usepackage{amsmath}
\usepackage{mathtools}

\usepackage{geometry} 
\usepackage{units}
\usepackage[binary-units=true]{siunitx}
\usepackage{pst-plot}
\usepackage{pgfplots}

% axis style
\pgfplotsset{every axis/.append style={
                    axis x line=middle,    % put the x axis in the middle
                    axis y line=middle,    % put the y axis in the middle
                    axis line style={<->}, % arrows on the axis
                    xlabel={$x$},          % default put x on x-axis
                    ylabel={$y$},          % default put y on y-axis
                    }}
% grid style
\pgfplotsset{
  grid style = {
    dash pattern = on 0.05mm off 1mm,
    line cap = round,
    black,
    line width = 0.5pt
  }
}
% line style
\pgfplotsset{myBlueStyle/.style={color=blue,no marks,line width=1pt,<->}} 
\pgfplotsset{myBlueStyle/.style={color=blue,no marks,line width=1pt,-}} 
\pgfplotsset{myBlueThickStyle/.style={color=blue,no marks,line width=2pt,-}} 
\pgfplotsset{myBlackThickStyle/.style={color=black,no marks,line width=2pt,-}} 

% arrow style: stealth stands for 'stealth fighter' 
\tikzset{>=stealth}

\DeclarePairedDelimiter\ceil{\lceil}{\rceil}
\DeclarePairedDelimiter\floor{\lfloor}{\rfloor}

\geometry{paperwidth=0.5in,paperheight=0.5in}  
\geometry{left=0.1in,right=0.1in}  
\geometry{top=0.1in,bottom=0.1in}  
\pagenumbering{gobble}

 \begin{document}
 \begin{flushleft}

\eject \pdfpagewidth=0.5in \pdfpageheight=0.75in
\begin{flalign*}
f(x) \nonumber
\end{flalign*}

\newpage
\eject \pdfpagewidth=1in \pdfpageheight=1in
\begin{flalign*}
v'=f(v,u,x) \nonumber \\
u'=f(v,u) \nonumber
\end{flalign*}

\newpage
\eject \pdfpagewidth=0.5in \pdfpageheight=0.75in
\begin{equation}
\begin{split}
\sum \nonumber
\end{split}
\end{equation}

\newpage
\eject \pdfpagewidth=0.5in \pdfpageheight=2.25in
\begin{equation}
\begin{split}
\begin{cases}
    {}\\           
    {}\\           
    {}\\           
    {}\\           
    {}\\           
    {}\\           
    {}\\           
    {}\\           
    {}\\           
\end{cases} \nonumber
\end{split}
\end{equation}

% Layer 0 weights

\newpage
\eject \pdfpagewidth=0.5in \pdfpageheight=0.75in
\begin{equation}
\begin{split}
w_{0,0}^0 \nonumber
\end{split}
\end{equation}

\newpage
\eject \pdfpagewidth=0.5in \pdfpageheight=0.75in
\begin{equation}
\begin{split}
w_{i,0}^0 \nonumber
\end{split}
\end{equation}

\newpage
\eject \pdfpagewidth=0.5in \pdfpageheight=0.75in
\begin{equation}
\begin{split}
w_{I,0}^0 \nonumber
\end{split}
\end{equation}

\newpage
\eject \pdfpagewidth=0.5in \pdfpageheight=0.75in
\begin{equation}
\begin{split}
w_{0,j}^0 \nonumber
\end{split}
\end{equation}

\newpage
\eject \pdfpagewidth=0.5in \pdfpageheight=0.75in
\begin{equation}
\begin{split}
w_{i,j}^0 \nonumber
\end{split}
\end{equation}

\newpage
\eject \pdfpagewidth=0.5in \pdfpageheight=0.75in
\begin{equation}
\begin{split}
w_{I,j}^0 \nonumber
\end{split}
\end{equation}

\newpage
\eject \pdfpagewidth=0.5in \pdfpageheight=0.75in
\begin{equation}
\begin{split}
w_{0,J}^0 \nonumber
\end{split}
\end{equation}

\newpage
\eject \pdfpagewidth=0.5in \pdfpageheight=0.75in
\begin{equation}
\begin{split}
w_{i,J}^0 \nonumber
\end{split}
\end{equation}

\newpage
\eject \pdfpagewidth=0.5in \pdfpageheight=0.75in
\begin{equation}
\begin{split}
w_{I,J}^0 \nonumber
\end{split}
\end{equation}

% Layer 1 weights
\newpage
\eject \pdfpagewidth=0.5in \pdfpageheight=0.75in
\begin{equation}
\begin{split}
w_{0,0}^1 \nonumber
\end{split}
\end{equation}

\newpage
\eject \pdfpagewidth=0.5in \pdfpageheight=0.75in
\begin{equation}
\begin{split}
w_{l,0}^1 \nonumber
\end{split}
\end{equation}

\newpage
\eject \pdfpagewidth=0.5in \pdfpageheight=0.75in
\begin{equation}
\begin{split}
w_{L,0}^1 \nonumber
\end{split}
\end{equation}

\newpage
\eject \pdfpagewidth=0.5in \pdfpageheight=0.75in
\begin{equation}
\begin{split}
w_{0,m}^1 \nonumber
\end{split}
\end{equation}

\newpage
\eject \pdfpagewidth=0.5in \pdfpageheight=0.75in
\begin{equation}
\begin{split}
w_{l,m}^1 \nonumber
\end{split}
\end{equation}

\newpage
\eject \pdfpagewidth=0.5in \pdfpageheight=0.75in
\begin{equation}
\begin{split}
w_{L,m}^1 \nonumber
\end{split}
\end{equation}

\newpage
\eject \pdfpagewidth=0.5in \pdfpageheight=0.75in
\begin{equation}
\begin{split}
w_{0,M}^1 \nonumber
\end{split}
\end{equation}

\newpage
\eject \pdfpagewidth=0.5in \pdfpageheight=0.75in
\begin{equation}
\begin{split}
w_{l,M}^1 \nonumber
\end{split}
\end{equation}

\newpage
\eject \pdfpagewidth=0.5in \pdfpageheight=0.75in
\begin{equation}
\begin{split}
w_{L,M}^1 \nonumber
\end{split}
\end{equation}


% Layer 0 Delays

\newpage
\eject \pdfpagewidth=0.5in \pdfpageheight=0.75in
\begin{equation}
\begin{split}
d_{0,0}^0 \nonumber
\end{split}
\end{equation}

\newpage
\eject \pdfpagewidth=0.5in \pdfpageheight=0.75in
\begin{equation}
\begin{split}
d_{i,0}^0 \nonumber
\end{split}
\end{equation}

\newpage
\eject \pdfpagewidth=0.5in \pdfpageheight=0.75in
\begin{equation}
\begin{split}
d_{I,0}^0 \nonumber
\end{split}
\end{equation}

\newpage
\eject \pdfpagewidth=0.5in \pdfpageheight=0.75in
\begin{equation}
\begin{split}
d_{0,j}^0 \nonumber
\end{split}
\end{equation}

\newpage
\eject \pdfpagewidth=0.5in \pdfpageheight=0.75in
\begin{equation}
\begin{split}
d_{i,j}^0 \nonumber
\end{split}
\end{equation}

\newpage
\eject \pdfpagewidth=0.5in \pdfpageheight=0.75in
\begin{equation}
\begin{split}
d_{I,j}^0 \nonumber
\end{split}
\end{equation}

\newpage
\eject \pdfpagewidth=0.5in \pdfpageheight=0.75in
\begin{equation}
\begin{split}
d_{0,J}^0 \nonumber
\end{split}
\end{equation}

\newpage
\eject \pdfpagewidth=0.5in \pdfpageheight=0.75in
\begin{equation}
\begin{split}
d_{i,J}^0 \nonumber
\end{split}
\end{equation}

\newpage
\eject \pdfpagewidth=0.5in \pdfpageheight=0.75in
\begin{equation}
\begin{split}
d_{I,J}^0 \nonumber
\end{split}
\end{equation}

% Layer 1 Delays
\newpage
\eject \pdfpagewidth=0.5in \pdfpageheight=0.75in
\begin{equation}
\begin{split}
d_{0,0}^1 \nonumber
\end{split}
\end{equation}

\newpage
\eject \pdfpagewidth=0.5in \pdfpageheight=0.75in
\begin{equation}
\begin{split}
d_{l,0}^1 \nonumber
\end{split}
\end{equation}

\newpage
\eject \pdfpagewidth=0.5in \pdfpageheight=0.75in
\begin{equation}
\begin{split}
d_{L,0}^1 \nonumber
\end{split}
\end{equation}

\newpage
\eject \pdfpagewidth=0.5in \pdfpageheight=0.75in
\begin{equation}
\begin{split}
d_{0,m}^1 \nonumber
\end{split}
\end{equation}

\newpage
\eject \pdfpagewidth=0.5in \pdfpageheight=0.75in
\begin{equation}
\begin{split}
d_{l,m}^1 \nonumber
\end{split}
\end{equation}

\newpage
\eject \pdfpagewidth=0.5in \pdfpageheight=0.75in
\begin{equation}
\begin{split}
d_{L,m}^1 \nonumber
\end{split}
\end{equation}

\newpage
\eject \pdfpagewidth=0.5in \pdfpageheight=0.75in
\begin{equation}
\begin{split}
d_{0,M}^1 \nonumber
\end{split}
\end{equation}

\newpage
\eject \pdfpagewidth=0.5in \pdfpageheight=0.75in
\begin{equation}
\begin{split}
d_{l,M}^1 \nonumber
\end{split}
\end{equation}

\newpage
\eject \pdfpagewidth=0.5in \pdfpageheight=0.75in
\begin{equation}
\begin{split}
d_{L,M}^1 \nonumber
\end{split}
\end{equation}

%
\newpage
\eject \pdfpagewidth=1.5in \pdfpageheight=1in
\begin{equation}
\label{eq:ReLuMax}
\begin{split}
\text{ReLu} &= \max{(0,x)} \nonumber
\end{split}
\end{equation}

\newpage
\eject \pdfpagewidth=2in \pdfpageheight=1in
\begin{equation}
\label{eq:ReLuCases}
\begin{split}
f(x)=
\begin{cases}
    0,      &\text{if $x < 0$;}\\           
    x,       &\text{otherwise. \nonumber}             
\end{cases}
\end{split}
\end{equation}

\newpage
\eject \pdfpagewidth=1.5in \pdfpageheight=1in
\begin{equation}
\label{eq:Sigmoid}
\begin{split}
f(x)= \frac{1}{1+e^{-t}} \nonumber
\end{split}
\end{equation}

\newpage
\eject \pdfpagewidth=5in \pdfpageheight=3in
\begin{tikzpicture}
    \begin{axis}[
        title={Sigmoid},
        %grid = major,
        xmin=-6,xmax=6,
        ymin=0,ymax=1,
        legend style={draw=none, fill=white},      
        legend pos= outer north east
        ]
        \addplot[myBlueThickStyle] expression[domain=-6:6,samples=100]{1/(1+e^(-x))} 
                    node at (axis cs:3,0.7){}; 
        \legend{{\large $f(x)=\frac{1}{1+e^{-t}}$}}
    \end{axis}
\end{tikzpicture}


\newpage
\eject \pdfpagewidth=6in \pdfpageheight=3in
\begin{tikzpicture}
    \begin{axis}[
        title={ReLu},
        xmin=-1,xmax=1,
        ymin=0,ymax=1,
        legend style={draw=none, fill=white},      
        legend pos= outer north east
        ]
        \addplot[myBlueThickStyle] expression[domain=0:1,samples=100]{x} node at (axis cs:0.6,0.15){}; 
         \addplot[myBlueThickStyle] expression[domain=-1:0,samples=100]{0};       
         \legend  {{ $f(x)= \begin{cases} 0,      &\text{if $x < 0$;}\\   x,       &\text{otherwise.}  \end{cases}$}}  
    \end{axis}
\end{tikzpicture}

\newpage
\eject \pdfpagewidth=6in \pdfpageheight=3in
\begin{tikzpicture}
    \begin{axis}[
        title={Saturation},
        xmin=-2,xmax=2,
        ymin=-1.5,ymax=1.5,
        legend style={draw=none, fill=white},      
        legend pos= outer north east
        ]
        \addplot[myBlueThickStyle] expression[domain=-1:1,samples=100]{x} node at (axis cs:0.6,0.15){}; 
         \addplot[myBlueThickStyle] expression[domain=-2:-1,samples=100]{-1};       
         \addplot[myBlueThickStyle] expression[domain=1:2,samples=100]{1};       
         \legend  {{ $f(x)= \begin{cases} 1,      &\text{if $x \ge 1$;}\\ x,      &\text{if $-1<x<1$;}\\  -1,      &\text{if $x \le -1$;} \end{cases}$}}  
    \end{axis}
\end{tikzpicture}

\newpage
\eject \pdfpagewidth=3in \pdfpageheight=1.5in
\begin{equation}
\label{eq:Izhikevich}
\begin{split}
&v' = 0.04v^2+5v + 140 - u - I\\
&u' = a(bv-u)  \\
&\text{if } v\ge  \SI{30}{\mV}, \text{ then } 
\begin{cases}
    v \leftarrow c\\           
    u \leftarrow u+d\\           
\end{cases} \nonumber
\end{split}
\end{equation}


\newpage
\eject \pdfpagewidth=3in \pdfpageheight=1.5in
\begin{equation}
\begin{split}
&\text{Izhikevich Model} \\
&v' = 0.04v^2+5v + 140 - u - I\\
&u' = a(bv-u)  \\
&\text{if } v\ge  \SI{30}{\mV}, \text{ then } 
\begin{cases}
    v \leftarrow c\\           
    u \leftarrow u+d\\           
\end{cases} \nonumber
\end{split}
\end{equation}



\newpage
\eject \pdfpagewidth=8in \pdfpageheight=1in
\begin{equation}
\label{eq:e1}
\begin{split}
f(x) \overset{ \text{def} }{=} x^{2} - 1 = 
\overbrace{a + b + \cdots + z}^{n}
\end{split}
\end{equation}

\begin{equation}
\label{eq:e2}
\begin{split}
f(x)=
\begin{cases}
    -x^{2},      &\text{if $x < 0$;}\\           
    \alpha + x,  &\text{if $0 \leq x \leq 1$;}\\
    x^{2},       &\text{otherwise.}             
\end{cases}
\end{split}
\end{equation}


 \end{flushleft}
 \end{document}
  
 
