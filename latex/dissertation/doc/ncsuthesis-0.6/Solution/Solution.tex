
\section[The Solution]{The Solution}
\label{sec:The Solution}

This work assumes that to support all types of disparate \ac{ann}s, the system needs to be able to operate directly from the \ac{dram}.
\iffalse 
This is because SRAM-based solutions assume memory reuse when processing a \ac{ann}.
However, when \ac{ann}s do not provide sufficient reuse these solutions become \ac{dram} bandwidth bound. 
\fi

Considering \ac{dram} is required to meet the main storage requirements of usefully sized \ac{ann}s, if an implementation can ensure the \ac{dram} bandwidth can meet the system requirements, why use \ac{sram} as an intermediate memory and waste the significant silicon area it consumes?

The question becomes, can an implementation employ \ac{dram} with minimal \ac{sram} and meet the system requirements?

This work's implementation operates directly out of \ac{dram}, but not just \ac{dram}, \ac{3ddram}.
This work has designed a system that can stay within the physical footprint of the \ac{3ddram} and thus can leverage the benefits of 3DIC.
The benefits of \ac{3dic}, which are reviewed in Chapter \ref{sec:3dic} include reduced energy, reduced area and increased connectivity and bandwidth.

\iffalse
Therefore, this work is able to propose a custom 3D-\ac{dram} that exposes more of the \ac{dram}s internal page and thus generates interface bandwidth that is on the order of 64 times that of the standard \ac{3ddram}.
\fi

% ----------------------------------------------------------------------------------------------------
% ----------------------------------------------------------------------------------------------------
Given the problem description \iffalse outlined in Section \ref{sec:The Problem},\fi the primary design considerations that drove the architecture of this work are :
\begin{outline}
  \1 \ac{dram} is required for storage of \ac{ann} parameters 
  \1 Target applications are unable to take advantage of memory reuse opportunities and therefore not able to achieve high performance using local \ac{sram} \iffalse to store \ac{ann} parameters or the \ac{ann} input \fi
  \1 Target applications will likely apply many disparate \acp{ann} to perform various system functions
  \1 Target applications will have space and power limitations
\end{outline}

When performing inference in \acp{ann}, the computational hotspot is the \ac{an} pre-synaptic summation shown in Figure \ref{fig:Rate Based Model} and \eqref{eq:activation function}.
This \ac{an} summation involves hundreds or thousands of multiply-accumulates of the pre-synaptic \ac{an} activations and corresponding connection weights. 
In this work, the \ac{an} activations and weights are stored in \ac{dram} with minimal local \ac{sram}. 
Therefore, because of the complex access protocol associated with \ac{dram}, one of the main objectives is to demonstrate the \ac{3ddram} can be accessed while maintaining the required average bandwidth to the processing elements.
\iffalse with relatively high levels of bus efficiency. \fi

The system has to process thousands of \acp{an} concurrently and do this with minimal unused bus cycles.
Therefore, the system must decode instructions, configure the various functions, pre-fetch and pipeline \ac{dram} data and perform the actual activation calculation. 

To maximize the processing bandwidth, these operations are all performed concurrently enabling this work to demonstrate the ability to meet and exceed the required \iffalse \SI[per-mode=symbol]{32}{\tera \bit \per \second} of\fi processing bandwidth as shown in \eqref{eq:maximumBandwidth}.

\iffalse
The problem associated with processing \acp{ann} are outlined in Section \ref{sec:The Problem}. 
\fi

\section[Novelty]{Novelty}
\label{sec:Novelty}

The novelty of this work includes:
\begin{outline}
    \1 An extensible architecture that can simultaneously process multiple disparate real-time \ac{ann}s 
      \2 with low power and real estate demands
    \1 Proposing a custom \ac{3ddram} providing a \textasciitilde 64X bandwidth benefit compared to standard \ac{3ddram}
      \2 the \ac{3ddram} could be employed in other applications
    \1 An \ac{ann} system that employs pure \ac{3dic} technology
      \2 providing power and performance benefits of remaining within a \ac{3dic} stack
    \1 Custom instructions and data structures that facilitate operating directly out of \ac{3ddram} 
      \2 maximizing processing bandwidth by ensuring effective use of the \ac{3ddram}
      \2 instruction format allow system functions to operate concurrently 
\end{outline}


\section[Summary]{Summary}
\label{sec:Summary}

This research explores a \ac{3dic} solution using a custom organized \ac{3ddram} in conjunction with unique data structures and custom processing modules to significantly reduce the 
area and power footprint of an application that needs to support the processing associated with multiple \ac{ann}s.
This work's system will provide at or near real-time performance required for systems employing multiple disparate \ac{ann}s while staying within acceptable area and power limits and will provide greater than an order of magnitude benefit over comparable solutions.

There will always be questions regarding the suitability of this work's target application, the baseline \ac{ann} and the \ac{binary32} number format, but these should be mitigated by providing an extensible architecture.
Given different processing and/or number format requirements, with reasonable modifications this work could provide a solution to most \ac{ann} system requirements.

\hfill %\break \newline

\iffalse
An overview of \ac{3dic} technology is given in Chapter \ref{sec:3dic}.
An overview on the pros and cons of \ac{dram} and \ac{sram} along with some proposed \ac{dram} customizations are given in Chapter \ref{sec:DRAM Customizations}.
\fi
Some state-of-the-art implementations are reviewed in Chapter \ref{sec:State of the art}.
An overview of the proposed system is described in Chapter \ref{sec:System Overview} with more details in Chapter \ref{sec:Detailed System Description}.
An overview of the instruction architecture is given in Chapter \ref{sec:System Operations}.
Simulation results are shown in Chapter \ref{sec:Results}.
The conclusion and further work are discussed in Chapter \ref{sec:Conclusions and Future Work}.
