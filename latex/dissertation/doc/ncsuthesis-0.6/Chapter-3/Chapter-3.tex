%% Lee
%% In dissertation, change 
%    section* to chapter 
%    subsection* to section
%    subsubsection* to subsection



\chapter{Research Objective and Anticipated Results}
%\section{Research Objective and Anticipated Results}
\label{chap-three}

The goals of this work is to research and design an ASIC based solution implemented using 3DIC technology able to support at or near real-time simultaneous 
processing of multiple ANNs.

This research will target a family of ANNs that have demonstrated efficacy in
a family of popular applications. These applications have large neural networks requiring a large amount of memory 
to support weight storage and the requirement of processing multiple ANNs at or near real-time.
% FIXME - should I use this line or a version of it
%As mentioned in section \ref{sec:chap-one}, it is our belief that most useful applications of ANNs fall into this category.

\vspace{-3mm}
\subsection*{Targeted Applications}
\label{sec:Targeted Applications}

\vspace{-2mm}
This work will target:
\begin{outline}
\renewcommand{\outlinei}{enumerate}
  \vspace{-3mm}
  \1 Brain-state-in-a-Box (BsB) and Cogent Confabulation (CC) because of their demonstrated efficacy in text recognition \cite{qiu2013parallel}.
    \vspace{-3mm}
    \2 The system described in \cite{qiu2013parallel} utilized 78 clusters with each cluster incorporating 22 Sony PS3 game stations and two NVidia GPU
       cards. The text recognition system was segmented into character recognition performed by the BsB networks and the word and sentence construction was
       performed by Cogent Confabulation.
       There is opportunity to accelerate BsB and potential to accelerate portions of the Cogent Confabulation networks.
  \vspace{-08mm}
  \1 Deep neural network classifiers \cite{krizhevsky2012imagenet} because of their demonstrated efficacy in image recognition 
    \vspace{-3mm}
    \2 The application space includes autonomous navigation of automobiles \cite{bojarski2016end}
       with a need to process multiple classifiers simultaneously at real-time.
  \vspace{-3mm}
  \1 Reinforcement learning in the context of ANNs
    \vspace{-3mm}
    \2 The reinforcement learning algorithm employs deep neural networks in action-value function approximation \cite{mnih2013playing}. 
       Considering most environments will have very large state spaces,
       environment exploration will result in large numbers of action-value calculations and therefore acceleration will improve processing times.
       It should be noted that supporting reinforcement learning may require provisions for acceleration of back-propagation.
\end{outline}
 The result of this research will be:
\vspace{-0.5mm}
\begin{outline}
\renewcommand{\outlinei}{enumerate}
  \vspace{-3mm}
  \1 a custom organized DRAM specification with optimal bank, page and IO organization to support the target systems
  \vspace{-3mm}
  \1 a set of data structures for storage of inputs, weights and matrices to ensure high bandwidth utilization
  \vspace{-3mm}
  \1 a 3DIC architecture including a communication bus proposal to carry inputs and results between processing layers and the memory management layer
    \vspace{-3mm}
    \2 will include a packet structure with data and configuration formats
%5 Dissertation
%%   \vspace{-3mm}
%%   \1 a management layer including:
%%     \vspace{-3mm}
%%     \2 a cluster CPU for general management
%%       \vspace{-1mm}
%%       \3 will include a "standard" interface for transferring data from and to a host system
%%     \vspace{-1mm}
%%     \2 a sequencer to control the sequence of operations associated with the algorithms 
%%       \vspace{-1mm}
%%       \3 control reading DRAM and transferring packets to and from the stack bus
%%       \vspace{-1mm}
%%       \3 will most likely be controlled by the cluster CPU
   \vspace{-3mm}
   \1 a processing layer with a group of streaming operations customized for each of the target ANNs 
    \vspace{-3mm}
    \2 the PE will include a SIMD engine to perform additional processing not suited to hardware acceleration
    \vspace{-1mm}
    \2 a separate processing layer(s) may be provided for each target ANN
\end{outline}
\vspace{-2mm}
This work expects to demonstrate a \iffalse greater than \else \fi power/performance improvement over a GPU solution while \iftrue significantly \fi reducing
the physical footprint.
The work also expects to demonstrate \iffalse significant \fi capacity and processing improvements compared to state-of-the-art ASIC solutions
which in most cases are not able to support ANNs of this type, number and size.

Current GPU's are to some extent general purpose processors and therefore have excess silicon and power not directly involved in solving
any given application.
It is not uncommon for a GPU, because of memory limitations to support a single effective ANN whilst consuming the order of $> \SI{100}{\W}$.
This work expects to demonstrate a \iffalse solution with a 20X\fi power/performance improvement over current GPU solutions by benefiting from the power and performance 
advantages of 3DIC architecture \iffalse dissipating approximately \SI{40}{\W}\fi
whilst simultaneouly supporting multiple ANNs.
%% FIXME:  Dissertation ??
%% This work also expects to maintain an overall power and space improvement over next generation GPU's that may utilize 3D-DRAM technology such as HBM.
%% This will be achieved by providing a high capacity system able to provide a system solution whilst remaining inside the 3D footprint and
%% avoiding the lower bandwidth and higher power associated with connectivity outside of the 3D footprint.

%% A secondary goal of this work may be to consider issues associated with connecting more than one of these 3D modules into a larger network.
%% This work may consider the impact of partitioning a network across multiple devices and issues associated with communicating neuron state information
%% between modules.

