

%% Lee
%% In dissertation, change section* to chapter and subsection* to section


\chapter{Motivation}
\label{Motivation}


\section[The Problem]{The Problem}
\label{sec:The Problem}

As mentioned in chapter \ref{sec:Introduction}, this work focuses on edge applications employing disparate \ac{ann}s and therefore assumes there are limited opportunities for both weight reuse and batch processing.

Given the storage requirements shown in table \ref{tab:Bandwidth and Storage Design Requirements}, it is generally accepted that \ac{dram} is required to store the \ac{ann} parameters.

When considering systems that will employ multiple \ac{dnn}s simultaneously suggests that these edge systems will require usable memory bandwidth of the order of 10s of \SI[per-mode=symbol]{}{\tera \bit \per \second} \eqref{eq:maximumBandwidth}.

In these cases, \textbf{\textcolor{black}{\ac{dram} bandwidth is the bottleneck}}.

\iffalse
Given the bandwidth and storage requirements shown in table \ref{tab:Bandwidth and Storage Design Requirements}, the problem becomes \hyphenquote{american}{\textbf{\textcolor{black}{to provide deterministic at or near real-time performance within tolerable power and space constraints for edge systems employing inference on multiple disparate useful-sized neural networks.}}}
%\hlc[gray]{hello}
\fi


\subsection{Why not \ac{sram}?}
\label{sec:Why not SRAM}

Why is it that much of the \ac{asic} and \ac{asip} \ac{ann} research employs \ac{sram} as an intermediate store? 

In practice there are benefits if the processing elements can operate out of \ac{sram}.
Certainly good performance and potentially low power.

Perhaps also because its easy to use? 

When compared to \ac{dram}, \ac{sram} has low latency. Also, the \ac{dram} access protocol is much more complicated to implement than \ac{sram}. 

Given that \ac{dram} is used for the main memory storage, having the processing elements operate out of \ac{sram} requires that the high cost of transferring data from the \ac{dram} to the \ac{sram} be absorbed by using that data multiple times or "reused".

But using \ac{sram} for intermediate storage makes assumptions on the type of \acp{ann} that can be supported and the application in which the \ac{ann} is being deployed.
The primary requirement of the type of \ac{ann} and the deployed application to allow effective use of \ac{sram} is "reuse", so once parameters are transferred and stored in \ac{sram}, these parameters can be reused such that the \ac{sram} isn't simply an intermediate memory but something akin to a cache.

In some \ac{ann}s there are reuse opportunities. 
A prime example is \acp{cnn}, where the connection weights are reused. In \acp{cnn}, common feature filters are passed across an input to form the next layer. 
These filter "kernels" can be held in memory and the input is read from \ac{dram} thus reducing the \ac{dram} bandwidth.
Even with \ac{dnn}s where different filters may be used for different \acp{roi} some filter reuse may be available.
Another form of reuse is in cloud applications or in training where there is opportunity to reuse inputs whilst performing batch processing.

But \ac{sram} comes at a price, its big. Often when we see physical layouts of \ac{ann} processors, they are dominated by the silicon area of the \ac{sram}. 
Because of the relatively large area required for \ac{sram}, companies attempt to create custom \acp{sram} to minimize the area impact.

So \ac{asic} and \ac{asip} \ac{ann} implementations that target applications that have considerable weight reuse and/or batch processing opportunities can effectively use \ac{sram} as an intermediate store.

But to reiterate, this work assumes the target application have limited or no opportunities for weight reuse or batch processing.

\iffalse
So the question becomes, can a system employ \ac{dram} with minimal \ac{sram} and still meet the system requirements?
\fi

\iffalse
We believe a system can be designed with \ac{dram} as the primary processing store. This will require careful use of data structures to describe storage within \ac{dram} to ensure we make good use of the potential bandwidth. But there are other benefits we will take advantage of, but more about that later.
\fi

\iffalse
There important application is disparate \ac{ann}s because specifically a form of \ac{dnn}, Convolutional Neural networks (\ac{cnn}) have gotten good press recently, but they are not the only \ac{dnn}.
\fi



\iffalse
So considering the performance improvements observed in other applications, it is expected that many customer facing or edge applications will implement multiple instances of artificial neural networks to perform various functions.
have very large memory and processing requirements.
require multiple instances of \ac{ann}s of similar size to the \ac{ann} described in \cite{krizhevsky2012imagenet}.

For example employing multiple cameras or monitoring and controlling different systems in a drone, a automobile each with an image recognition \ac{ann}\cite{krizhevsky2012imagenet}\cite{bojarski2016end} for navigation, engine monitoring along with other system control.
\fi

\section{Alternatives}
\label{sec:Alternatives}

\subsection{\Acfp{gpu}}
\label{sec:gpu}
Some might suggest the requirements of these applications would be satisified by employing multiple \acp{gpu}.
In practice, \acp{gpu} are used to implement large \ac{ann}s and in some \ac{ann} architectures, such as \acp{cnn}, they are quite effective. However, we should not forget they are a) not optimized purely for \ac{ann} processing, b) are restricted by available SRAM and c) they are power hungry. 
These limitations will limit the effectiveness of \acp{gpu} regardless of what we might hear from the \ac{gpu} community.
Even in the case of newer \acp{gpu} which are employing 2.5DIC technology, the memory bandwidth will still be limited by available \ac{dram} technology.
For example, a 2.5D solution employing High bandwidth Memory (HBM) would be limited to a maximum raw bandwith of the order of \SI[per-mode=symbol]{6}{\tera \bit \per \second} \cite{Nvidia_p100_summary_datasheet}.
Also, it has proven very difficult, if not impossible to take advantage of the available memory bandwidth \cite{farabet2011neuflow} \cite{tensorflow2015-whitepaper}.

A solution could employ multiple devices, but there would be significant power and real-estate issues. 
The typical high performance \ac{gpu} consumes between the order of \SI{100}{\watt} and \SI{200}{\watt}.
A multiple \ac{gpu} implementation would have a high real-estate impact and a system power approaching a \SI[per-mode=symbol]{1}{\kilo \watt}.

Overall \acp{gpu} have limited suitability to meet this works target application requirements.


\subsection{\Acp{asic}/\Acp{asip}}
\label{sec:asicAndAsip}
Much of the \ac{ann} application specific (\ac{asic}/\ac{asip}) research has focused on taking advantage of the performance and ease of use of \acf{sram}.
These implementations can be shown to be effective with specific \ac{ann} architectures (e.g. \ \ac{cnn}), server applications or the "toy examples" but when a system requires multiple disparate \ac{ann}s in an edge application, \textbf{\textcolor{black}{existing implementations do not provide the required flexibility, storage capacity and deterministic performance}}.

Even in cloud applications, there are limitations on reuse. We paraphrase a quote from a Google paper \cite{tensorflow2015-whitepaper} on their Tensor Processing Unit ASIC (TPU):

\hyphenquote{american}{the architecture research community is paying attention to \acp{ann}, but of all the papers at ISCA 2016 on hardware accelerators for \acp{ann}, alas, all nine papers looked at \ac{cnn}s, and only two mentioned other \acp{ann}. 
Unfortunately \ac{cnn}s represent only about 5\% of our datacenter NN workload}

The applications targeted by the google TPU \cite{tensorflow2015-whitepaper} assume multiple requests, so reuse in the form of batch processing is still of great benefit, but the bulk of the requests in \cite{tensorflow2015-whitepaper} are fully-connected \ac{dnn}s and in these cases weight reuse is not as benefitial and the performance of the TPU is degraded when implementing these fully-connected \ac{dnn}s.

Therefore, implementations that focus on \ac{cnn}s can suffer from severe degradation in performance when targeting generic types of \ac{ann}, such as locally and fully connected \ac{dnn}s and LSTMs.

Considering this work focuses on edge applications employing disparate \ac{ann}s and assumes both weight reuse and batch processing do not apply, regardless of how implementations employ \ac{sram} as an intermediate store, \textbf{\textcolor{black}{\ac{dram} bandwidth is the bottleneck}}.



\iffalse How this work addresses the problem are outlined in section \ref{chap-five}. \fi


\section[The Solution]{The Solution}
\label{sec:The Solution}

It is generally accepted that realistically, \ac{dram} is required to meet the main storage requirements of useful sized \ac{ann}s.

This work also believes that to support all types of disparate \ac{ann}s, the system needs to be able to operate directly from the \ac{dram}.
\iffalse 
This is because SRAM-based solutions assume memory reuse when processing a \ac{ann}.
However, when \ac{ann}s do not provide sufficient reuse these solutions become \ac{dram} bandwidth bound. 
\fi

If an implementation can ensure the \ac{dram} bandwidth can meet the system requirements, why use \ac{sram} as an intermediate memory and waste the significant silicon area they consume.

The question becomes, can an implementation employ \ac{dram} with minimal \ac{sram} and meet the system requirements?

This works implementation operates directly out of \ac{dram}, but not just \ac{dram}, \ac{3ddram}.
This work has designed a system that can stay within the physical footprint of the \ac{3ddram} and thus can leverage the benefits of 3DIC.
The benefits of \ac{3dic}, which are reviewed in chapter \ref{sec:3dic} include reduced energy, reduced area and increased connectivity and bandwidth.

\iffalse
Therefore, this work is able to propose a custom 3D-\ac{dram} that exposes more of the \ac{dram}s internal page and thus generates interface bandwidth that is of the order of 64 times that of the standard \ac{3ddram}.
\fi

% ----------------------------------------------------------------------------------------------------
% ----------------------------------------------------------------------------------------------------
Given the problem description \iffalse outlined in section \ref{sec:The Problem},\fi the primary design considerations that drove the architecture of this work are :
\begin{outline}
  \1 \ac{dram} is required for storage of \ac{ann} parameters 
  \1 Target applications are unable to take advantage of memory reuse opportunities and therefore not able to achieve high performance using local \ac{sram} \iffalse to store \ac{ann} parameters or the \ac{ann} input \fi
  \1 Target application will likely apply many disparate \acp{ann} to perform various system functions
  \1 Target application will have space and power limitations
\end{outline}

When performing inference in \acp{ann}, the computational hotspot is the \ac{an} pre-synaptic summation shown in figure \ref{fig:Rate Based Model} and equation \eqref{eq:activation function}.
This \ac{an} summation involves hundreds or thousands of multiply-accumulates of the pre-synaptic \ac{an} activations and corresponding connection weights. 
In this work, the \ac{an} activations and weights are stored in \ac{dram} with minimal local \ac{sram}. 
Therefore, because of the complex access protocol associated with \ac{dram}, one of the main objectives is to demonstrate the \ac{3ddram} can be accessed while maintaining the required average bandwidth to the processing elements.
\iffalse with relatively high levels of bus efficiency. \fi

The system has to process thousands of \acp{an} concurrently and do this with minimal unused bus cycles.
Therefore, the system must decode instructions, configure the various functions, pre-fetch and pipeline \ac{dram} data and perform the actual activation calculation. 

To maximize the processing bandwidth, these operations are all performed concurrently enabling this work to demonstrate the ability to meet and exceed the required \iffalse \SI[per-mode=symbol]{32}{\tera \bit \per \second} of\fi processing bandwidth as shown in equation \eqref{eq:maximumBandwidth}.

\iffalse
The problem associated with processing \acp{ann} are outlined in section \ref{sec:The Problem}. 
\fi

\subsection[Novelty]{Novelty}
\label{sec:Novelty}

The novelty of this work includes:
\begin{outline}
    \1 A system that can simultaneously process multiple disparate \ac{ann}s at or near real-time 
      \2 with low power and real-estate demands
    \1 A custom \ac{3ddram} providing a \textasciitilde 64X bandwidth benefit compared to standard \ac{3ddram}
      \2 the \ac{3ddram} could be employed in other applications
    \1 A system that employs pure \ac{3dic} technology
      \2 providing power and performance benefits of remaining within a \ac{3dic} stack
    \1 Custom instructions and data structures that facilitate operating directly out of \ac{3ddram} 
      \2 maximizing processing bandwidth by ensuring effective use of the \ac{3ddram}
      \2 instruction format allow system functions to operate concurrently 
\end{outline}


\subsection[Summary]{Summary}
\label{sec:Summary}

This research explores a \ac{3dic} solution using a custom organized \ac{3ddram} in conjunction with unique data structures and custom processing modules to significantly reduce the 
area and power footprint of an application that needs to support the processing associated with multiple \ac{ann}s.
This works system will provide at or near real-time performance required for systems employing multiple disparate \ac{ann}s whilst staying within acceptable area and power limits and will provide greater than an order of magnitude benefit over comparable solutions.

An overview of \ac{3dic} technology is given in chapter \ref{sec:3dic}.
An overview on the pros and cons of \ac{dram} and \ac{sram} along with some proposed \ac{dram} customizations are given in chapter \ref{sec:DRAM Customizations}.
Some state-of-the-art implementations are reviewed in chapter \ref{sec:State of the art}.
An overview of the proposed system is described in chapter \ref{sec:System Overview} with more details in chapter \ref{sec:Detailed System Description}.
An overview of the instruction architecture is given in chapter \ref{sec:System Operations}.
Simulation results are shown in chapter \ref{sec:Results}.
The conclusion and further work are discussed in chapter \ref{sec:Conclusions and Future Work}.
