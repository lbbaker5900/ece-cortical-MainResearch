%% ------------------------------ Abstract ---------------------------------- %%
\begin{abstract}

%\lipsum[1-6]

\iffalse
This dissertation explores employing a customized \ac{3ddram} and a pure \ac{3dic} system in the acceleration of \acp{ann} for systems deployed in embedded applications.
Assuming \acs{ann}s fulfill their potential, it is this work's belief that these systems will employ \acs{ann}s for various functions, such as engine monitoring, anomaly detection, navigation etc. and that the various system functions are implemented with a set of disparate \acs{ann}s.
\fi
%The combination of an embedded system and multiple disparate \acp{ann} means systems that employ \ac{sram} will not be able to provide the neccessary performance in terms of power or capacity.
\iffalse
A further assumption is that these embedded systems do not have access to cloud servers.

Although \acp{ann} have been known of for many decades, it hasn't been until the last few years that they have demonstrated efficacy in applications such as image recognition and voice recognition.

\acp{an} take their inspiration from neuron behavior observed in the mammalian brain, although implementations are simplifications of what actually exists in the brain.
These simplifications range from attempts to emulate the actual spiking behavior of real neurons to \acp{an} that simply encode the spiking behavior in the form of a number or rate.

The \ac{ann}s that have demonstrated most efficacy are a family of neural networks that can be described as \acp{dnn}.
These \acp{dnn} are created by cascading layers of rate-based \acp{an} to form a large, layered \ac{ann} employing ten's of thousands or more of \acp{an}. 

Considering the storage required for the input and the \ac{ann} parameters, the storage requirements result in gigabytes of memory.
When these \acp{ann} are required to be solved in fractions of a second, the processing and memory bandwidth becomes prohibitive.
Unfortunately, to achieve a high performance, existing implementations rely on processing a batch of inputs, such as processing a batch of images or voice recordings which all use the same \ac{ann} or by employing a sub-family of \acp{dnn}, known as \acp{cnn}, which reuse portions of the \ac{ann} parameters.

These techniques allow these implementations to hold and reuse data in fast local \ac{sram}.
%With this work's target application, the assumption is there is little opportunity for batch processing or reuse therefore data must be drawn constantly from main memory, which generally is \ac{dram}.
The assumption is that embedded systems do not have batch processing opportunities or parameter reuse therefore data must be drawn constantly from main memory.

\fi

%The combination of an embedded system and multiple disparate \acp{ann} means systems that employ \ac{sram} will not be able to provide the neccessary performance in terms of power or capacity.
%One area of integrated circuit technology that has started to be employed in \ac{ann}s is \acp{3dic}.  \acp{3dic} have the potential to increase connectivity, and thus bandwidth, 
%and keep power dissipation within acceptable levels.
This work demonstrates how a customized \ac{3ddram} with a very wide data-bus can be combined with application-specific layers to produce a system meeting the requirements of embedded systems employing multiple instances of disparate \acp{ann}.
This work avoids any dependencies on \ac{sram} that might limit the size of supported \acp{ann}. 
This work demonstrates the required utilization of the very wide \ac{dram} data-bus when employing instructions and data structures that facilitate operating directly out of the \ac{3ddram}.
By allowing system functions to operate asynchronously this work is able to absorb the latencies associated with \ac{dram} and provide the bandwidth required to support multiple useful sized disparate \acp{ann}.
By demonstrating effective use of the very-wide bus of a customized \ac{3ddram}, 
this system demonstrates a 3X power improvement and 6X area improvement over similar \ac{ann} systems.

%By eliminating the dependency on \ac{sram} and proposing customizations to an existing \ac{3ddram} to provide a very wide data-bus, 
%this system is able to implement most useful types of \acp{dnn} and provide a 3X power improvement and 6X area improvement over similar \ac{ann} systems.

\end{abstract}


%% ---------------------------- Copyright page ------------------------------ %%
%% Comment the next line if you don't want the copyright page included.
\makecopyrightpage

%% -------------------------------- Title page ------------------------------ %%
\maketitlepage

%% -------------------------------- Dedication ------------------------------ %%
\begin{dedication}
 \centering To my wife Mandy, my children Adam, Rachel and Paul and my parents Joan and Barry.
\end{dedication}

%% -------------------------------- Biography ------------------------------- %%
\begin{biography}
The author was born in the United Kingdom.  After high school he took a job in a local electronic engineering firm
under a vocational program.
While working on the manufacturing floor and seeing the white coated "engineers" being called down from upstairs to solve the "big" problems, he decided he wanted to wear one
of those white coats.
The journey to the "white coat" took him to Brighton Polytechnic, now Brighton University and a First Class Honors Degree in Electrical Engineering.
After working in the UK for a couple of years, he moved to the United States.
The journey included a family with a daughter and two sons.
The education continued with a Masters in Engineering from Villanova University and a Masters in Business Administration from North
Carolina State University.

With the family now being somewhat independent, he decided to make a career change that would hopefully include teaching.

That career change included enrolling in the Electrical Engineering PhD program at North
Carolina State University. This stage of the education journey has resulted in this dissertation.

%\begin{chapquote}{Lewis Carroll, \textit{Alice in Wonderland}}
%``Begin at the beginning,'' the King said, gravely, ``and go on till you
%come to an end; then stop.''
%\end{chapquote}
%\setlength{\epigraphwidth}{6in} 
%\epigraphfontsize{\small\itshape}
%\epigraph{do not stand still and do not let your past dictate your future}{--- \textup{Unknown}}
%\epigraph{do not stand }{--- \textup{Unknown}}

%\begin{savequote}[10cm]
%---do not stand still
%\qauthor{Unknown}
%---do not let your past dictate your future
%\qauthor{Unknown}
%Cookies! Give me some cookies!
%\qauthor{Cookie Monster}
%\end{savequote}
%\begin{displayquote}

%\end{displayquote}
%\textcquote{do not stand still.}
%\textcquote{do not let your past dictate your future.}
%\blockquote{unknown}{do not let your past dictate your future.}
%\hyphenquote{american}{quote snippet in a different language}

Remember:

"do not stand still."  

"do not let your past dictate your future."

%"`foo"'

\end{biography}

%% ----------------------------- Acknowledgements --------------------------- %%
\begin{acknowledgements}
At a personal level, I would like to thank my wife Mandy and my children Adam, Rachel and Paul for their encouragement.

I would like to thank my advisor, Paul Franzon for his help in making this possible.

I would also like to thank Steve Lipa, Jong Beom Park and Josh Schabel for their feedback on this dissertation.

I would also like to thank my fellow students, especially Jong Beom, Josh, Sumon and Weifu for their healthy discussions and, being an older student, referring to me as Lee and not Sir or Mr. Baker.

This work was funded in part by DARPA and AFRL under FA8650-15-1-7518 and DARPA and ONR under N00014-17-1-3013, as part of the CHIPS program.
\end{acknowledgements}


%% Lee
% Uncomment for dissertation (change to iftrue)
%\iffalse
\iftrue

\thesistableofcontents

\thesislistoftables

\thesislistoffigures

\fi
