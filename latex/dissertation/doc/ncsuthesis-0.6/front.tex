%% ------------------------------ Abstract ---------------------------------- %%
\begin{abstract}

%\lipsum[1-6]

This dissertation explores employing a \ac{3dic} in the acceleration of \acp{ann} for systems deployed in embedded applications.
Assuming \acs{ann}s fulfill their potential, it is this work's belief that these systems will employ \acs{ann}s for various functions, such as engine monitoring, anomaly detection, navigation etc. and that the various system functions are implemented with a set of disparate \acs{ann}s.
A further assumption is that these embedded systems may not have access to cloud servers or the cloud servers do not provide the necessary turn-around time for processing the \acs{ann}.

Although \acp{ann} have been known of for many decades, it hasn't been until the last few years that they have demonstrated efficacy in applications such as image recognition and voice recognition.
\iffalse
These \acp{ann} have demonstrated significant improvements over what was considered to be state-of-the-art algorithms.
\fi

\iffalse
In most cases, \ac{ann}s must be "trained" to perform their function and these recent breakthoughs can, in a large part 
be attributed to the availability of "big" training data.

This data takes the form of images, recordings etc. and this along with the available storage capacity means that the required training data is now available 
to train these \ac{ann}s.
\fi

\acp{an} take their inspiration from neuron behavior observed in the mammalian brain, although implementations are simplifications of what actually exists in the brain.
These simplifications range from attempts to emulate the actual spiking behavior of real neurons to \acp{an} that simply encode the spiking behavior in the form of a number or rate.

\iffalse
Surprisingly, the \ac{ann}s that have demonstrated the most efficacy are those that employ the more simpler rate-based \acp{an}.
Although there is a belief that the more complex spiking neurons have the ability to outperform the simpler rate-based neurons, this work focuses on the proven efficacy of \ac{ann}s formed from rate-based \acp{an}.
\fi

\iffalse
Now this may be in large part because these simpler rate-based ANs are easier to process or because large \acp{ann} employing the more complex spiking \acp{an} have yet to
demonsrate improved efficacy in the various applications.
There is a belief that the more complex spiking neurons have the ability to outperform the simpler rate-based neurons but this work focuses on
the proven efficacy of \ac{ann}s formed from rate-based ANs.
\fi

The \ac{ann}s that have demonstrated most efficacy are a family of neural networks that can be described as \acp{dnn}.
These \acp{dnn} are created by cascading layers of rate-based \acp{an} to form a large, layered \ac{ann} employing ten's of thousands or more of \acp{an}. 

Considering the storage required for the input and the \ac{ann} parameters, the storage requirements result in gigabytes of memory.
When these \acp{ann} are required to be solved in fractions of a second, the processing and memory bandwidth becomes prohibitive.

Unfortunately, to achieve a high performance, existing implementations rely on processing a batch of inputs, such as processing a batch of images or voice recordings which all use the same \ac{ann} or by employing a sub-family of \acp{dnn}, known as \acp{cnn}, which reuse portions of the \ac{ann} parameters.
These techniques allow these implementations to hold and reuse data in fast local \ac{sram}.
%With this work's target application, the assumption is there is little opportunity for batch processing or reuse therefore data must be drawn constantly from main memory, which generally is \ac{dram}.
The assumption is that embedded systems do not have batch processing opportunities or parameter reuse therefore data must be drawn constantly from main memory.


\iffalse
These \acp{ann} are in most cases generate outputs in the form of a classification, such as the probability of an image containing a certain object
or the output of some approximated function, which might be the expected cost from making a stock trade.

Now researchers have experimented with various sized NNs for various applcations, but those that have demonstrated the most efficacy employ ten's of thousands
of \acp{an}.

In many implementations of these useful sized NNs, the performance is impacted by the memory bandwidth of the system. 
Much of the NN application specific (ASIC/ASIP) research has focused on taking advantage of the performance and ease of use of Static Random Access Memory or SRAM. 
These implementations can be shown to be effective with specific NN architectures, such as Convolutional NNs but in reality, these implementations do not provide the flexibility, 
storage capacity and deterministic performance required to implement all useful sized NNs.

In addition, it is this works belief that that real-world applications will employ multiple instances of these useful sized \ac{ann}s and current implementations will not
meet the demands of these multi NN systems.
\fi

\iffalse
Researchers have attempted to solve this with
the use of fast local SRAM. These implementations rely on the \ac{ann} being able to store data in the local SRAM and reuse that data without having to
constantly go to the main memory, which is often implemented in slower DRAM.
However, many DNNs do not exhibit this ability to reuse the contents of local SRAM, or not to the extent that makes these implementations acceptable.
Some work has included combining SRAM with DRAM but these implementations still exhibit poor performance with a wider family of \ac{ann}s.
\fi

One area of integrated circuit technology that hasn't been widely used in \ac{ann}s is \acp{3dic}.  \acp{3dic} have the potential to increase connectivity, and thus bandwidth, 
and keep power dissipation within acceptable levels.

This work demonstrates how a customized \ac{3dic} \ac{dram} can be combined with application-specific layers to produce a system meeting the required level of performance in systems with multiple instances of disparate \acp{ann}.
\iffalse
This work includes utilizing a customized 3D DRAM along with a system which includes a management layer which coordinates and executes \ac{ann} operations in the form
of unique instructions and a processing layer able to process data from the DRAM, via the manager at a bandwidth that meets the demands of a system employing
multiple \ac{ann}s.
\fi



\end{abstract}


%% ---------------------------- Copyright page ------------------------------ %%
%% Comment the next line if you don't want the copyright page included.
\makecopyrightpage

%% -------------------------------- Title page ------------------------------ %%
\maketitlepage

%% -------------------------------- Dedication ------------------------------ %%
\begin{dedication}
 \centering To my wife Mandy, my children Adam, Rachel and Paul and my parents Joan and Barry.
\end{dedication}

%% -------------------------------- Biography ------------------------------- %%
\begin{biography}
The author was born in a the United Kingdom.  After high school he took a job in a local electronic engineering firm
under a vocational program.
While working on the manufacturing floor and seeing the white coated "engineers" being called down from upstairs to solve the "big" problems, he decided he wanted to wear one
of those white coats.
The journey to the "white coat" took him to Brighton Polytechnic, now Brighton University and a First Class Honours Degree in Electrical Engineering.
After working in the UK for a couple of years, he moved to the United States.
The journey included a family with a daughter and two sons.
The education continued with a Masters in Engineering from Villanova University and a Masters in Business Administration from North
Carolina State University.

With the family now being somewhat independent, he decided to make a career change that would hopefully include teaching.

That career change included enrolling in the Electrical Engineering PhD program at North
Carolina State University. This stage of the education journey has resulted in this dissertation.

%\begin{chapquote}{Lewis Carroll, \textit{Alice in Wonderland}}
%``Begin at the beginning,'' the King said, gravely, ``and go on till you
%come to an end; then stop.''
%\end{chapquote}
%\setlength{\epigraphwidth}{6in} 
%\epigraphfontsize{\small\itshape}
%\epigraph{do not stand still and do not let your past dictate your future}{--- \textup{Unknown}}
%\epigraph{do not stand }{--- \textup{Unknown}}

%\begin{savequote}[10cm]
%---do not stand still
%\qauthor{Unknown}
%---do not let your past dictate your future
%\qauthor{Unknown}
%Cookies! Give me some cookies!
%\qauthor{Cookie Monster}
%\end{savequote}
%\begin{displayquote}

%\end{displayquote}
%\textcquote{do not stand still.}
%\textcquote{do not let your past dictate your future.}
%\blockquote{unknown}{do not let your past dictate your future.}
%\hyphenquote{american}{quote snippet in a different language}

Remember:

"do not stand still."  

"do not let your past dictate your future."

%"`foo"'

\end{biography}

%% ----------------------------- Acknowledgements --------------------------- %%
\begin{acknowledgements}
At a personal level, I would like to thank my wife Mandy and my children Adam, Rachel and Paul for their encouragement.

I would like to thank my advisor, Paul Franzon for his help in making this possible.

I would also like to thank Steve Lipa, Jong Beom Park and Josh Schabel for their feedback on this dissertation.

I would also like to thank my fellow students, especially Jong Beom, Josh, Sumon and Weifu for their healthy discussions and, being an older student, referring to me as Lee and not Sir or Mr. Baker.

This work was funded in part by DARPA and AFRL under FA8650-15-1-7518 and DARPA and ONR under N00014-17-1-3013, as part of the CHIPS program.
\end{acknowledgements}


%% Lee
% Uncomment for dissertation (change to iftrue)
%\iffalse
\iftrue

\thesistableofcontents

\thesislistoftables

\thesislistoffigures

\fi
