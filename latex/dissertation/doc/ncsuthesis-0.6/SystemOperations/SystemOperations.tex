

\chapter{System Operations}
\label{sec:System Operations}

As mentioned in Chapter \ref{sec:System Overview}, an \ac{ssc} has 32 execution lanes allowing the simultaneous processing of up to 32 \acp{an}.
When processing a group of \acp{an} the basic operations to determine their states are:
\begin{outline}
  %\lbbcleanspace{20pt}
  \lbbcleanspace
  \1 manager streams the states of the pre-synaptic \acp{an} to the \ac{pe}
  \1 manager streams the weights of the pre-synaptic connections to the \ac{pe}
  \1 each execution lane in the \ac{pe} operates directly on the two argument streams using the \ac{stop} block
  \1 the \ac{pe} \ac{simd} block takes the 32 results from the \ac{stop} block and performs the activation function to generate the \ac{an} states
  \1 the \ac{pe} \ac{simd} packetizes the \ac{an} states and sends the packet to the manager
  \1 the manager replicates the \ac{an} state data over the \ac{noc} to any dependent \acp{ssc}
  \1 if the local \ac{ssc} is dependent on the result, the manager saves the \ac{an} state data in local \ac{ssc} \ac{dram}
\end{outline}

Although the primary focus of this work is effective use of a novel \ac{3ddram} along with an expansive system to process \acp{ann}, a complete system needs to also provide support features for tasks such as downloading the \ac{ann} parameters to memory from a host system and a host system also needs to download new inputs and upload \ac{ann} outputs.
This work has created a system and an infrastructure to include these support tasks.
Although not all of the features required for an actual final product have been included in this work's design and verification effort,  the infrastructure required to easily add the missing features has been provided.

This work has developed an instruction architecture to support and describe the operations associated with the processing of an \ac{ann} along with the support tasks required when employing this work in a product implementation.

\iffalse

The manager's primary responsibility is to decode instructions and descriptors and :

\begin{outline}
    \1 Instruction decode
    \1 Internal Configuration messaging
    \1 Operand read
    \1 Result write
    \1 Host to \ac{ssc} and \ac{ssc} to \ac{ssc} communication
\end{outline}


The \ac{pe} has three major blocks:

\begin{outline}
    \1 Streaming operation function (\ac{stop})
      \2 Processes data from the manager on-the-fly without storing in local \ac{sram}
    \1 \ac{simd}
      \2 processes the data from the \ac{stop} before returning data to the manager
    \1 DMA/local memory controller
      \2 transfer configuration data to \ac{pe} controller or to store \ac{stop} results to a small local \ac{sram} which can be used for access by \ac{simd} or by the \ac{stop} function
\end{outline}

\fi
% ----------------------------------------------------------------------------------------------------
% ----------------------------------------------------------------------------------------------------
\section{Instructions}
\label{sec:Instructions}

An instruction is coarse grained in as much as it provides the information to perform all the operations associated with a high level task.
The manager is responsible for instruction decode and coordinating the various data flows and configuration of the modules throughout the \ac{ssc}.
The \ac{pe} is responsible for the main algorithm operations using a combination of its \ac{stop} and \ac{simd} blocks.
The information in the instruction provides the information to control these finer grained functions within the manager and the \ac{pe} along with the various system support tasks.

To provide the fine grained information, an instruction is partitioned into sub-instructions called descriptors.
An instruction can contain one or more descriptors and each of these descriptors contains the information to control a specific operation(s) within the \ac{ssc} to perform the high level task.
For example, to process a group of \acp{an}, the instruction contains a descriptor to communicate where the pre-synaptic \ac{an} states are stored, another to communicate where the connection weights are stored, another describes what activation function should be used and another where in memory the resulting \ac{an} states should be stored.

% ----------------------------------------------------------------------------------------------------
\subsection{Instruction Types}
\label{sec:Instruction Types}
There are two instruction types currently defined, a \textbf{{\textcolor{black}{configuration}}} instruction and a \textbf{\textcolor{black}{compute}} instruction.
The configuration instruction has been defined to deal with synchronization and data downloads and uploads which includes \ac{ann} parameters and input, \ac{ssc} instructions, \ac{simd} and \ac{stop} operation pointers and \ac{simd} instructions.
The compute instruction has been defined to deal with computing the states of a group of \acp{an}. \par
Although this work's focus has been on the compute instruction -- as it has the most influence on system performance -- configuration instructions have been defined and tested to provide an extensible system.
\iffalse
\begin{outline}
  \lbbcleanspace
   \1 \ac{ann} parameters and input
   \1 \ac{ssc} instructions
   \1 \ac{simd} and \ac{stop} operation pointers
   \1 \ac{simd} instructions
\end{outline}
Typically an instruction contains information to process a group of \acp{an} but there are other instruction types to synchronize.
A group can be anywhere from one to 32 \acp{an} and is based on the number of execution lanes in the \ac{ssc} (see Section \ref{sec:Processing a group of ANes}) and how the user partitions the \ac{ann} across the available \acp{ssc}.
\fi

A generic instruction is an n-tuple where the tuple elements are descriptors and the number of descriptors can vary based on the high level task being performed. 
These descriptors are decoded and used to configure the various functions in the \ac{ssc} that will take part in completing the instruction. 
The contents of a single descriptor may be sent to multiple functions and in some cases the manager doesn't even parse the contents of the descriptor but immediately passes it to a dependent function.
This allows the system to concurrently prepare for the tasks involved with the instruction.


\subsection{Compute Instruction}
\label{sec:Compute Instruction}

The compute instruction typically contains four descriptors for configuring the tasks associated with processing a group of \acp{an}.
The instruction can be seen in Figure \ref{fig:Instruction (4-tuple example)} and includes: 
\begin{itemize}
  %\lbbcleanspace
  \item Operation descriptor containing:
    \begin{itemize}
      \item \ac{stop} operation
      \item \ac{simd} operation
      \item Number of active lanes
      \item Operand Vector length
    \end{itemize}
  \item Two memory read descriptors containing:
    \begin{itemize}
        \item addresses for the pre-synaptic \ac{an} states and connection weights for the two argument streams to the \ac{pe}
        \item read data to execution lane multiplex method (broadcast/vectored)
    \end{itemize}
  \item Memory write descriptor containing:
    \begin{itemize}
      \item \ac{dram} address for \ac{an} states
    \end{itemize}
\end{itemize}
\iffalse
\begin{figure}[!t]
\centering
\captionsetup{justification=centering}
\captionsetup{width=.9\linewidth}
\centerline{
\mbox{\includegraphics[width=.9\linewidth]{instruction4Tuple}}
}
\caption{Typical compute instruction (4-tuple)}
\label{fig:Instruction (4-tuple example)}
\end{figure}
\fi
\begin{figure}[!t]
  % the [] contains position info e.g. [!t] means here
  \centering
    \captionsetup{justification=centering, skip=10pt}
    \vspace{10mm}
    \begin{bytefield}[bitwidth=0.49em, endianness=big]{80}
      %\bitheader{0,32,64,96,128} \\
      %\begin{rightwordgroup}{\scriptsize Option Cycle}
        \colorbitbox{desc!40}{20}{\scriptsize Operation descriptor\\ \vspace{-0em}(MAC+ReLu)} & \colorbitbox{desc!40}{20}{\scriptsize Read descriptor\\ \vspace{-0em}(Input ROI)} & \colorbitbox{desc!40}{20}{\scriptsize Read descriptor \\ \vspace{-0em}(pre-synaptic \ac{an} states)} & \colorbitbox{desc!40}{20}{\scriptsize Write descriptor\\ \vspace{-0em}(store \ac{an} states)} 
      %\end{rightwordgroup}  \\
    \end{bytefield}
\caption{Typical compute instruction (4-tuple)}
\label{fig:Instruction (4-tuple example)}
\end{figure}


The descriptor also employs an n-tuple format where the elements contain configuration options required by the operation.
The option elements within a descriptor are a two-tuple with option and associated value and are referred to as option tuples.
These option tuples include a type and value which contain information such as storage descriptor pointer (see Section \ref{sec:Storage Descriptor}), \ac{pe} operations and the number of operands.
The length of the value field is currently eight bits or 24-bits. The 24-bit value field is referred to as an extended tuple and is currently used for memory address, number of operands and configuration options.
In Figure \ref{fig:descriptorTuple} we see the format of a 5-tuple operation descriptor and a list of option types is shown in Table \ref{tab:Option tuple functions}.
%It should be noted that not all option tuple type are currently used in the system but were provided for expansibility.

\iffalse
\begin{figure}[!t]
\centering
\captionsetup{justification=centering}
\captionsetup{width=.9\linewidth}
\centerline{
\mbox{\includegraphics[width=.9\linewidth]{descriptorTuple}}
}
\caption{Operation descriptor (5-tuple example)}
\label{fig:descriptorTuple}
\end{figure}
\fi


\begin{figure}[!t]
  % the [] contains position info e.g. [!t] means here
  \centering
  \captionsetup{justification=centering}

  \begin{minipage}{1\textwidth}
    \centering
    \begin{minipage}{0.85\textwidth}
      \centering
      \captionsetup{justification=centering}
      \captionsetup{width=.9\linewidth}
      \centerline{
      \mbox{\includegraphics[width=.9\linewidth]{descriptorTuple}}
      }
      \vspace{-2mm}
      \caption{Operation descriptor (5-tuple example)}
      \label{fig:descriptorTuple}
    \end{minipage}
    \begin{minipage}{0.85\textwidth}
        \vspace{5mm}
        \begin{adjustbox}{width=1\textwidth}
            \footnotesize
            \begin{tabular}{ |c|c|c|c|  }
              \hline
              \rowcolor{gray!50}
              \multicolumn{4}{|c|}{Source} \\
              \hline
              \rowcolor{gray!25}
              Type & Type Code & extd &  Value Description  \\
              \hline
              NOP                              &    0   &  N&  \acl{nop} \\
              source                           &    1   &  N&  Used to define the source of any data, such as memory or \ac{pe}  \\
              target                           &    2   &  N&  Used to define the target for any data, such as memory or \ac{pe}  \\
              transfer type                    &    3   &  N&  How data will be directed, vector or broadcast                    \\
              number of lanes                  &    4   &  N&  number of active execution lanes used in operation \\
              \ac{stop} pointer                &    5   &  N&  pointer to \ac{pe} \ac{stop} operation table in \ac{pe} controller \\
              \ac{simd} pointer                &    6   &  N&  pointer to \ac{pe} \ac{simd} instruction memory  \\
              Memory storage descriptor        &    7   &  Y&  pointer to storage descriptor used in memory read or write \\
              num of arg0 operands             &    8   &  Y&  number of operands sent to execution lane stream 0\\
              num of arg1 operands             &    9   &  Y&  number of operands sent to execution lane stream 1\\
              num of return packets            &    10  &  Y&  number of response packets generated by PE \\
              config sync                      &    11  &  Y&  Synchronization \\
              config data                      &    12  &  Y&  Data transfers\\
              status                           &    13  &  Y&  status information \\
              \hline
            \end{tabular}
        \end{adjustbox}
        \caption{Option tuple functions}
        \label{tab:Option tuple functions}
    \end{minipage}
  \end{minipage}
\end{figure}


To pull it all together, Figure \ref{fig:Instruction Details} demonstrates a four-tuple compute instruction with details shown for each of the descriptors.
Figure \ref{fig:Instruction and Descriptors} shows the compute instruction which contains three 4-tuple and one 5-tuple descriptors.
The memory write descriptor shows two storage option elements which indicates the resulting \ac{an} states need to be saved in the memory of two \acp{ssc}.
The waveform in Figure \ref{fig:Instruction memory waveform} is from the verification environment and shows the instruction as it is read out of the manager's instruction memory.
The waveform also shows an example of the common interface signaling described in Section \label{sec:Common Bus Signalling} with the signals \texttt{wum\textunderscore\textunderscore wud\textunderscore\textunderscore icntl} and \texttt{wum\textunderscore\textunderscore wud\textunderscore\textunderscore dcntl} being used to delineate the instruction and descriptors respectively.
As can be seen in Figure \ref{fig:Instruction memory waveform}, the instruction memory transfers three descriptor elements per cycle shown on the bus signals \texttt{wum\textunderscore\textunderscore wud\textunderscore\textunderscore option\textunderscore type} and \texttt{wum\textunderscore\textunderscore wud\textunderscore\textunderscore option\textunderscore value}.
\\
Note: The signal name convention used between blocks in the \ac{rtl} is <src>\textunderscore\textunderscore <dest>\textunderscore\textunderscore <signal\textunderscore\textunderscore  name>. In Figure \ref{fig:Instruction memory waveform},~ \texttt{wum} and \texttt{wud} refer to ``work unit memory'' and ``work unit decode''
which correspond to the manager instruction memory and instruction decoder respectively.

\begin{figure}
\centering
  \begin{subfigure}{.95\textwidth}
    \centering
    \mbox{\includegraphics[width=1\linewidth]{instructionAndDescriptors}}
    \captionsetup{justification=centering, skip=6pt}
    \caption{Compute instruction and descriptors}
    \label{fig:Instruction and Descriptors}
  \end{subfigure}%

\bigskip

  \vspace{-35pt}
  \begin{subfigure}{1\textwidth}
    \centering
    \vspace{40pt}
    \includegraphics[width=0.95\textwidth]{instructionWaveform}
    \captionsetup{justification=centering, skip=10pt}
    \caption{Instruction memory waveform}
    \label{fig:Instruction memory waveform}
  \end{subfigure}%
%\vspace{-10pt}
\captionsetup{justification=centering, skip=16pt}
\caption{Compute Instruction details}
\label{fig:Instruction Details}
\end{figure}



% ----------------------------------------------------------------------------------------------------
\subsubsection{Accessing of Pre-synaptic \ac{an} states and connection weights}
\label{sec:AccessingANStates}

A part of the research is determining how to store the \ac{ann} input and parameters in such a way as to effectively make use of main \ac{dram} bandwidth. 

To provide parameters for the up to 32 execution lanes within the \ac{pe}, the \ac{an} parameters are stored in consecutive address locations. 
With one read to the \ac{dram}, we access 128 words. This provides four weights for each of the 32 \acp{an} being processed. 
These weights are sent to each lane of the \ac{pe} over four cycles. 
We will discuss memory efficiency later, but by taking advantage of the multiple \ac{dram} banks along with pre-fetching and buffering, we are able to make very efficient use of the available bandwidth.

Although \ac{an} parameters (weights) are stored in contiguous memory locations, providing the input state to a particular \ac{an} presents us with an interesting problem.

Most often DNNs are represented by layers of \acp{an} whose pre-synaptic neurons are from the previous layer. These previous layers represent the input to a given layer. The first layer's input is the actual input to the \ac{ann}.

The input can be represented in the form of a 2-D array of \ac{an} states. For the sake of generality, the input array elements are considered as \ac{an} states.

Any given \ac{an} operates on a \ac{roi} within the input array.

\subsubsection{Storage Descriptor}
\label{sec:Storage Descriptor}

Figure \ref{fig:roiStorage} shows an input to a \ac{ann} layer in the form of a 2-D array along with the \ac{roi} of two \acp{an}.

\begin{figure}[!t]
\centering
\captionsetup{justification=centering}
\captionsetup{width=.9\linewidth}
\centerline{
\mbox{\includegraphics[width=.9\linewidth]{roiStorage.jpg}}
}
\caption{ROI Storage}
\label{fig:roiStorage}
\end{figure}

The various connection weights are stored in multiple contiguous sections. However, it's not possible to arrange the input in such a way that each \ac{an}'s \ac{roi} can be stored in contiguous memory locations. 
The \ac{roi} arrangement shown in Figure \ref{fig:roiStorage} is typical. Assuming the input array is stored in row-major order, an \ac{roi} is drawn from disjoint sections of memory. 
These disjoint sections contain a number of \ac{an} states, in this case 14, and the sections are separated by a gap of a number of memory addresses. 
When the parameters are accessed while performing a particular operation, the memory controller within the manager must be informed of the start address and the lengths of the sections and gaps. 
In practice groups of \acp{an} share a common \ac{roi}, so often when reading an \ac{roi} from the \ac{dram} it is broadcast across a group of execution lanes.

The read efficiency problem is solved by again taking advantage of the \ac{dram}s banks and pages.

This work proposes a data structure to describe these \ac{roi} storage locations.

Although disparate groups of \acp{an} may have different start addresses for their \acp{roi}, a commonality is observed in the \ac{roi} section lengths and gaps. So for each \ac{an} group, the group's \ac{roi} starting address is stored along with a pointer to a common set of section lengths/gaps. This structure is termed a storage descriptor.

This storage descriptor contains, among other things, the start address of the \ac{roi} and a pointer to a section/gap descriptor. Many storage descriptors point to a common section/gap descriptor. This avoids having to have a unique section/gap descriptor for each \ac{an} group.

Figure \ref{fig:storageDescriptor} shows the structure of the storage descriptor. The \ac{sod}, \ac{mod} and \ac{eod} are used to delineate each storage descriptor in memory.

\begin{figure}[!t]
\centering
\captionsetup{justification=centering}
\captionsetup{width=.9\linewidth}
\centerline{
\mbox{\includegraphics[width=.9\linewidth]{storageDesc.jpg}}
}
\caption{Storage Descriptor}
\label{fig:storageDescriptor}
\end{figure}

% ----------------------------------------------------------------------------------------------------
\subsubsection{Writing \ac{an} state results to memory}
\label{sec:writingANStates}

When the \ac{pe} has processed the group of \acp{an}, the new \ac{an} states are sent back to the manager and stored in \ac{dram} in the row-major array format described earlier.
\iffalse
A significant difference taken advantage of is that for any given operation, the system is writing far less than is being read. For example, the \ac{roi} and parameters are usually vectors that exceed 100 elements and in many cases many more. When an operation is complete, in almost all cases one word per lane is written back to main memory. 
Now that sounds like writing back has a very small impact on performance, but with \acp{dram} that's not always true.
\fi
When an operation is complete, in almost all cases one word per lane is written back to \ac{dram}.
Considering a \ac{dram} page contains 128 words, the system typically writes a quarter of a page and this is a relatively inefficient use of \ac{dram} bandwidth. 
However, the pre-synaptic fanin typically far exceeds 100 elements and in the baseline \ac{an} shown in Table \ref{tab:Baseline Layer Configuration} the average fanin is \num 1650.
So the read-to-write ratio is very high and the inefficient write has little impact on the overall performance.

As discussed in Section \ref{sec:Inter-Manager Communication}, in many cases the results have to be provided not only to the local \ac{ssc} \ac{dram} but also to other \acp{ssc} memory. 
This is handled by examining the write storage descriptors and if at least one storage descriptor address references another \ac{ssc}'s memory, all the write storage descriptors in the instruction are included in the \ac{noc} packet (see Figure \ref{fig:NoC packet format}).

% ----------------------------------------------------------------------------------------------------
% ----------------------------------------------------------------------------------------------------
\iffalse
\section{PE Operations}
\label{sec:PE Operations}

% ----------------------------------------------------------------------------------------------------
\subsection{Streaming Operations (\ac{stop})}
\label{sec:streamingOps}
The operations performed by the \ac{stop} are primarily multiple-accumulate with a transfer to the \ac{simd} or to local memory.

Even though the baseline system focuses on the \ac{an} multiply-accumulate followed by a ReLu activation function, the system has built in flexibility into the \ac{stop} function to allow other functions to be added

In most cases, the \ac{stop} module will operate on the \ac{an} state and weights provided by the manager and provide the result to the \ac{simd}.
% ----------------------------------------------------------------------------------------------------
\subsection{SIMD}
\label{sec:SIMD}

The \ac{simd} is a 32-lane processor with some built-in special functions, such as the ReLu operation.

The \ac{simd} will take the result provided by the \ac{stop} and perform a ReLu. The result will, in most cases, then transmitted back to the manager.

% ----------------------------------------------------------------------------------------------------
\subsection{Configuration}
\label{sec:peConfiguration}

To configure these operations, two pointers are sent to the \ac{pe}. These pointers index into a small local memory which provides a program counter (\ac{pc}) to the function to be performed by the \ac{simd} and a configuration entry for the operation to be performed by the \ac{stop}.

The \ac{pe} is able to perform its operation concurrently on 32 lanes. However, there are cases when less than 32 lanes will be employed. This may occur if the number of \acp{an} being processed is not modulo-32. In this case, the manager provides the number of lanes being processed for any given operation. In addition, the length of the vector of operands is also sent to the \ac{pe} by the manager.
\fi

\subsection{Configuration Instruction}
\label{sec:Configuration Instruction}

The configuration instruction is used for :
\begin{itemize}
  \lbbcleanspace
    \item Data transfer
    \begin{itemize}
      \item Download Instructions from host to \ac{ssc} manager instruction memory
      \item Download Sync group data from host to \ac{ssc} manager 
      \item Download \ac{ann} parameters and input from host to \ac{ssc} memory
      \item Upload \ac{ann} output to host from \ac{ssc} memory
    \end{itemize}
  \item System synchronization
    \begin{itemize}
      \item Send a sync message to a group of \acp{ssc}
      \item Wait for sync messages from a group of \acp{ssc} or host
      \item Pause instruction fetch
      \item Flush \ac{pe} operations
    \end{itemize}
\end{itemize}

The configuration instruction contains one descriptor and there are two configuration types which are characterized by the descriptor option tuple contents.
All configuration descriptors start with a configuration option tuple. There are two configuration tuple types, sync and data.
The sync and data option types are extended types that have a 24-bit option value.
The 24-bit option value is used to define one of up to eight mode registers, each of which defines the type of configuration operation and various options.
The configuration option tuple can be seen in Figure \ref{fig:Configuration tuple}.

The data transfer configuration instruction is shown in Figure \ref{fig:Data transfer instruction}. The descriptor is a 2-tuple with a configuration data option type and a storage option type.
The sync configuration instruction is shown in Figure \ref{fig:Sync instruction} and the descriptor is a 1-tuple with a sync configuration option type.

\begin{figure}[h]
\centering

  %\begin{minipage}{1\textwidth}
  %\begin{subfigure}{1\textwidth}
    \centering
    \captionsetup{justification=centering, skip=10pt}
    \begin{bytefield}[bitwidth=1.49em, endianness=big]{32}
      \bitheader{0,20,21,23,24,31} \\
      %\begin{rightwordgroup}{\scriptsize Option Cycle}
        \colorbitbox{optiontype!40}{8}{\scriptsize Sync or Data} & \colorbitbox{optionvalue!40}{3}{\scriptsize Mode Register ID} & \colorbitbox{optionvalue!40}{21}{\scriptsize Mode register value } 
      %\end{rightwordgroup}  \\
    \end{bytefield}
    \captionsetup{justification=centering, skip=9pt}
    \vspace{-2mm}
    \captionof{figure}{Configuration tuple}
    \label{fig:Configuration tuple}
  %\end{subfigure}%
  %\end{minipage}
\end{figure}

\begin{figure}[h]
\centering

  \begin{subfigure}{.85\textwidth}
    \centering
    \mbox{\includegraphics[width=1\linewidth]{dataTransferInstruction}}
    \captionsetup{justification=centering, skip=6pt}
    \caption{Data transfer instruction }
    \label{fig:Data transfer instruction}
  \end{subfigure}%

\bigskip

  \vspace{-35pt}
  \begin{subfigure}{0.85\textwidth}
     \centering
     \vspace{40pt}
     \includegraphics[width=1\textwidth]{syncInstruction}
     \captionsetup{justification=centering, skip=10pt}
     \caption{Sync instruction }
     \label{fig:Sync instruction}
  \end{subfigure}%
  %\vspace{-10pt}
  \captionsetup{justification=centering, skip=16pt}
  \caption{Configuration instruction types}
  \label{fig:Configuration Instruction types}
\end{figure}

The data option value mode register \cite{standard2007jedec} defines the type of data transfer along with information to aid the transfer. 
The storage option type contains a storage descriptor pointer which specifies the address of memory transfers.
An overview of the configuration instructions is given in Sections \ref{sec:Data Transfer Instruction} and \ref{sec:Sync Instruction} with additional details in Section \ref{sec:Decoding Configuration Instructions}.

\subsubsection{Configuration Data Instruction}
\label{sec:Data Transfer Instruction}

There are currently four mode registers defined, an instruction download register, a sync group download register, a memory download register and a memory upload register.
The contents of the mode register specify the type of transfer, the size of the transfer and some additional flags.
When transferring to or from memory, an additional descriptor element contains a storage descriptor defining where the data should read or written.

\subsubsection{Configuration Sync Instruction}
\label{sec:Sync Instruction}

The option element is a sync option whose value contains a mode register.
There are currently four mode registers defined, a send, wait, pause and a flush register.
The contents of the send and wait mode register specify the group of \acp{ssc} to be synchronized. 
The send register causes a sync \ac{noc} packet to be sent to all \acp{ssc} in the group.
The wait register causes the manager to wait for a \ac{noc} sync packet to be received from all \acp{ssc} in the group.
The pause mode register causes the instruction fetch logic to pause the specified number of clock cycles.
The flush mode register causes the instruction fetch logic to wait for all outstanding \ac{pe} operations to be returned before continuing.

\subsection{Multiple Instruction Functions}
\label{sec:Multiple Instruction Functions}

When instructions or data are downloaded, in some cases there are tasks in the system that must be performed by chaining instructions together.
This is the case when downloading the \ac{pe} operation pointers and the \ac{simd} instructions.
In these cases, the host will have to first download the data to the \ac{ssc} local \ac{dram} in conjunction with a \ac{ssc} configuration download instruction.
The data is then transferred to the \ac{pe} using an operation instruction with the \ac{stop} configured as a \ac{nop} so the data will pass through the \ac{stop} with the small local \ac{sram} as the target. 
The \ac{pe} controller will then transfer the contents of the \ac{sram} to \ac{simd} instruction memory or the operation pointer memory.
As an example, loading the \ac{simd} instruction memory requires the procedure described in algorithm \ref{alg:Load SIMD Instruction memory}.

\begin{algorithm}
\caption{Load SIMD Instruction memory}
\label{alg:Load SIMD Instruction memory}
\begin{algorithmic}[1]
    %      inst       desc
    \State{// last compute instruction}
    \State \makebox[1.05cm][r]{COM:[}\makebox[0.25cm][r]{[}\makebox[0.85cm][r]{  op}\makebox[0.15cm][l]{:}\makebox[0.15cm][r]{[}\makebox[1.00cm][r]{stOp:}\makebox[1.25cm][l]{fpmac}\makebox[0.25cm][r]{,}\makebox[1.00cm][r]{simd:}\makebox[0.80cm][l]{relu }\makebox[0.25cm][r]{,}\makebox[1.65cm][r]{numop0:}\makebox[0.75cm][l]{<n>}\makebox[0.25cm][r]{,}\makebox[1.65cm][r]{numop1:}\makebox[1.00cm][l]{<n>  }\makebox[0.25cm][r]{]} \\
           \makebox[1.05cm][r]{     }\makebox[0.25cm][r]{[}\makebox[0.85cm][r]{  mr}\makebox[0.15cm][l]{:}\makebox[0.15cm][r]{[}\makebox[1.00cm][r]{ tgt:}\makebox[1.25cm][l]{std0 }\makebox[0.25cm][r]{,}\makebox[1.00cm][r]{txfr:}\makebox[0.80cm][l]{bcast}\makebox[0.25cm][r]{,}\makebox[1.65cm][r]{ lanes:}\makebox[0.75cm][l]{<l>}\makebox[0.25cm][r]{,}\makebox[1.65cm][r]{  StoD:}\makebox[1.00cm][l]{<ptr>}\makebox[0.25cm][r]{]} \\
           \makebox[1.05cm][r]{     }\makebox[0.25cm][r]{[}\makebox[0.85cm][r]{  mr}\makebox[0.15cm][l]{:}\makebox[0.15cm][r]{[}\makebox[1.00cm][r]{ tgt:}\makebox[1.25cm][l]{std1 }\makebox[0.25cm][r]{,}\makebox[1.00cm][r]{txfr:}\makebox[0.80cm][l]{vec  }\makebox[0.25cm][r]{,}\makebox[1.65cm][r]{ lanes:}\makebox[0.75cm][l]{<l>}\makebox[0.25cm][r]{,}\makebox[1.65cm][r]{  StoD:}\makebox[1.00cm][l]{<ptr>}\makebox[0.25cm][r]{]} \\
           \makebox[1.05cm][r]{     }\makebox[0.25cm][r]{[}\makebox[0.85cm][r]{  mw}\makebox[0.15cm][l]{:}\makebox[0.15cm][r]{[}\makebox[1.00cm][r]{ src:}\makebox[1.25cm][l]{stu  }\makebox[0.25cm][r]{,}\makebox[1.00cm][r]{txfr:}\makebox[0.80cm][l]{vec  }\makebox[0.25cm][r]{,}\makebox[1.65cm][r]{ lanes:}\makebox[0.75cm][l]{<l>}\makebox[0.25cm][r]{,}\makebox[1.65cm][r]{  StoD:}\makebox[1.00cm][l]{<ptr>}\makebox[0.25cm][r]{,}\makebox[1.00cm][r]{  StoD:}\makebox[1.00cm][l]{<ptr>}\makebox[0.25cm][r]{]} 
\\
    \State{//---------------------------------------- Start download ---------------------------------------- }
    \State{// make sure all compute instruction are complete}
    \State \makebox[1.05cm][r]{CFG:[}\makebox[0.25cm][r]{[}\makebox[0.85cm][r]{ cfg}\makebox[0.15cm][l]{:}\makebox[0.15cm][r]{[}\makebox[1.00cm][r]{sync:}\makebox[0.25cm][r]{[}\makebox[1.75cm][r]{flush}\makebox[0.25cm][r]{]}\makebox[0.25cm][r]{]}
    \State \makebox[1.05cm][r]{CFG:[}\makebox[0.25cm][r]{[}\makebox[0.85cm][r]{ cfg}\makebox[0.15cm][l]{:}\makebox[0.15cm][r]{[}\makebox[1.00cm][r]{sync:}\makebox[0.25cm][r]{[}\makebox[1.75cm][r]{send:}\makebox[0.25cm][r]{[}\makebox[1.00cm][l]{host }\makebox[0.25cm][r]{]}\makebox[0.25cm][r]{]}\makebox[0.25cm][r]{]}
    \State \makebox[1.05cm][r]{CFG:[}\makebox[0.25cm][r]{[}\makebox[0.85cm][r]{ cfg}\makebox[0.15cm][l]{:}\makebox[0.15cm][r]{[}\makebox[1.00cm][r]{sync:}\makebox[0.25cm][r]{[}\makebox[1.75cm][r]{wait:}\makebox[0.25cm][r]{[}\makebox[1.00cm][l]{host }\makebox[0.25cm][r]{]}\makebox[0.25cm][r]{]}\makebox[0.25cm][r]{]}
    \State{// Host sends release}
\\
    \State{// Host starts simd instruction download to SSC memory}
    \State{// Next SSC instruction prepares wr\textunderscore cntl for data from Host}
    \State \makebox[1.05cm][r]{CFG:[}\makebox[0.25cm][r]{[}\makebox[0.85cm][r]{ cfg}\makebox[0.15cm][l]{:}\makebox[0.15cm][r]{[}\makebox[1.00cm][r]{data:}\makebox[0.25cm][r]{[}\makebox[1.75cm][r]{mem\textunderscore dn:}\makebox[1.00cm][l]{<m>}\makebox[0.25cm][r]{,}\makebox[1.65cm][r]{  StoD:}\makebox[1.00cm][l]{<ptr>}\makebox[0.25cm][r]{]}
    \State \makebox[1.05cm][r]{CFG:[}\makebox[0.25cm][r]{[}\makebox[0.85cm][r]{ cfg}\makebox[0.15cm][l]{:}\makebox[0.15cm][r]{[}\makebox[1.00cm][r]{sync:}\makebox[0.25cm][r]{[}\makebox[1.75cm][r]{pause:}\makebox[0.25cm][r]{[}\makebox[1.00cm][l]{ind }\makebox[0.25cm][r]{]}\makebox[0.25cm][r]{]}\makebox[0.25cm][r]{]}
    \State{// fetch paused waiting for release, wr\textunderscore cntl ready for Host data}
    \State{// wr\textunderscore cntl releases fetch when data transfer complete}
    \State \makebox[1.05cm][r]{COM:[}\makebox[0.25cm][r]{[}\makebox[0.85cm][r]{  op}\makebox[0.15cm][l]{:}\makebox[0.15cm][r]{[}\makebox[1.00cm][r]{stOp:}\makebox[1.25cm][l]{ld\textunderscore simd}\makebox[0.25cm][r]{,}\makebox[1.00cm][r]{simd:}\makebox[0.80cm][l]{nop  }\makebox[0.25cm][r]{,}\makebox[1.65cm][r]{numop0:}\makebox[0.75cm][l]{<m>}\makebox[0.25cm][r]{]} \\
           \makebox[1.05cm][r]{     }\makebox[0.25cm][r]{[}\makebox[0.85cm][r]{  mr}\makebox[0.15cm][l]{:}\makebox[0.15cm][r]{[}\makebox[1.00cm][r]{ tgt:}\makebox[1.25cm][l]{std0}\makebox[0.25cm][r]{,}\makebox[1.00cm][r]{txfr:}\makebox[0.80cm][l]{bcast}\makebox[0.25cm][r]{,}\makebox[1.65cm][r]{ lanes:}\makebox[0.75cm][l]{1}\makebox[0.25cm][r]{,}\makebox[1.65cm][r]{  StoD:}\makebox[1.00cm][l]{<ptr>}\makebox[0.25cm][r]{]}
    \State{// manager sending instruction data to PE using compute operation with NOPs}
    \State{// flush PE to ensure instruction data complete}
    \State \makebox[1.05cm][r]{CFG:[}\makebox[0.25cm][r]{[}\makebox[0.85cm][r]{ cfg}\makebox[0.15cm][l]{:}\makebox[0.15cm][r]{[}\makebox[1.00cm][r]{sync:}\makebox[0.25cm][r]{[}\makebox[1.75cm][r]{flush}\makebox[0.25cm][r]{]}\makebox[0.25cm][r]{]}
    %\State \makebox[1.05cm][r]{CFG:[}\makebox[0.25cm][r]{[}\makebox[0.85cm][r]{ cfg}\makebox[0.15cm][l]{:}\makebox[0.15cm][r]{[}\makebox[1.00cm][r]{sync:}\makebox[0.25cm][r]{[}\makebox[1.75cm][r]{send:}\makebox[0.25cm][r]{[}\makebox[1.00cm][l]{host }\makebox[0.25cm][r]{]}\makebox[0.25cm][r]{]}\makebox[0.25cm][r]{]}
    %\State \makebox[1.05cm][r]{CFG:[}\makebox[0.25cm][r]{[}\makebox[0.85cm][r]{ cfg}\makebox[0.15cm][l]{:}\makebox[0.15cm][r]{[}\makebox[1.00cm][r]{sync:}\makebox[0.25cm][r]{[}\makebox[1.75cm][r]{wait:}\makebox[0.25cm][r]{[}\makebox[1.00cm][l]{host }\makebox[0.25cm][r]{]}\makebox[0.25cm][r]{]}\makebox[0.25cm][r]{]}
    \State{//---------------------------------------- end of download ---------------------------------------- }
\\
    \State{// continue with compute instructions}
    \State \makebox[1.05cm][r]{COM:[}\makebox[0.25cm][r]{[}\makebox[0.85cm][r]{  op}\makebox[0.15cm][l]{:}\makebox[0.15cm][r]{[}\makebox[1.00cm][r]{stOp:}\makebox[1.25cm][l]{fpmac}\makebox[0.25cm][r]{,}\makebox[1.00cm][r]{simd:}\makebox[0.80cm][l]{relu }\makebox[0.25cm][r]{,}\makebox[1.65cm][r]{numop0:}\makebox[0.75cm][l]{<n>}\makebox[0.25cm][r]{,}\makebox[1.65cm][r]{numop1:}\makebox[1.00cm][l]{<n>  }\makebox[0.25cm][r]{]} \\
           \makebox[1.05cm][r]{     }\makebox[0.25cm][r]{[}\makebox[0.85cm][r]{  mr}\makebox[0.15cm][l]{:}\makebox[0.15cm][r]{[}\makebox[1.00cm][r]{ tgt:}\makebox[1.25cm][l]{std0 }\makebox[0.25cm][r]{,}\makebox[1.00cm][r]{txfr:}\makebox[0.80cm][l]{bcast}\makebox[0.25cm][r]{,}\makebox[1.65cm][r]{ lanes:}\makebox[0.75cm][l]{<l>}\makebox[0.25cm][r]{,}\makebox[1.65cm][r]{  StoD:}\makebox[1.00cm][l]{<ptr>}\makebox[0.25cm][r]{]} \\
           \makebox[1.05cm][r]{     }\makebox[0.25cm][r]{[}\makebox[0.85cm][r]{  mr}\makebox[0.15cm][l]{:}\makebox[0.15cm][r]{[}\makebox[1.00cm][r]{ tgt:}\makebox[1.25cm][l]{std1 }\makebox[0.25cm][r]{,}\makebox[1.00cm][r]{txfr:}\makebox[0.80cm][l]{vec  }\makebox[0.25cm][r]{,}\makebox[1.65cm][r]{ lanes:}\makebox[0.75cm][l]{<l>}\makebox[0.25cm][r]{,}\makebox[1.65cm][r]{  StoD:}\makebox[1.00cm][l]{<ptr>}\makebox[0.25cm][r]{]} \\
           \makebox[1.05cm][r]{     }\makebox[0.25cm][r]{[}\makebox[0.85cm][r]{  mw}\makebox[0.15cm][l]{:}\makebox[0.15cm][r]{[}\makebox[1.00cm][r]{ src:}\makebox[1.25cm][l]{stu  }\makebox[0.25cm][r]{,}\makebox[1.00cm][r]{txfr:}\makebox[0.80cm][l]{vec  }\makebox[0.25cm][r]{,}\makebox[1.65cm][r]{ lanes:}\makebox[0.75cm][l]{<l>}\makebox[0.25cm][r]{,}\makebox[1.65cm][r]{  StoD:}\makebox[1.00cm][l]{<ptr>}\makebox[0.25cm][r]{,}\makebox[1.00cm][r]{  StoD:}\makebox[1.00cm][l]{<ptr>}\makebox[0.25cm][r]{]} 
\end{algorithmic}
\end{algorithm}




% ----------------------------------------------------------------------------------------------------
% ----------------------------------------------------------------------------------------------------
\section{Host Instructions}
\label{sec:Host Instructions}

The host is responsible for transferring \ac{ann} parameters and input data to the \ac{ssc} and for transferring \ac{ann} output data back to the host.
It is also responsible for downloading configuration data including \ac{ssc} instructions, \ac{pe} and \ac{simd} instructions and configuration tables such as storage pointer tables and \ac{pe}, \ac{stop} and \ac{simd} operation tables.

The host will use the \ac{noc} to carry the commands and data to accomplish the transfers.
The commands will employ the same option tuple method as described in Section \ref{sec:Configuration Instruction}.

A transfer of data to the \ac{ssc} can be solicited or unsolicited.
In the unsolicited case, the host initiates the transfer by sending the first \ac{noc} packet with two option tuples along with the first data as shown in Figure \ref{fig:Host transfer configuration packet}.
This first packet contains a configuration data option tuple (see Figure \ref{fig:Configuration tuple}) which contains a mode register used to define the type of transfer.
The second tuple has a storage descriptor pointer defining where in memory to write the data.
Once the first packet has been received, the \ac{ssc} expects only to receive data packets as shown in Figure \ref{fig:Host transfer data packet}.
In the solicited case, the transfer is initiated from a configuration data transfer instruction in which case the main controller receives the mode register and storage descriptor from the instruction decoder.
An example instruction can be seen in Figure \ref{fig:Data transfer instruction}.
In this case, the host will only send data packets as shown in Figure \ref{fig:Host transfer data packet}.
Additional information on the configuration data option tuple mode registers can be found in Sections \ref{sec:Memory download reg} and \ref{sec:Memory upload reg}.

In the case of a data download, the main controller sends the storage tuple to the memory write controller and indicates it's a \ac{dma}.
This will inform the memory write controller to expect further data packets and associate those with the current storage descriptor.
In the case of a data upload, the main controller will check the status of the instruction decode and if it has not been halted by a sync instruction, it will halt the instruction decoder.
The main controller then sends the storage tuple to the lane 0 memory read controller.

The data to be transferred is included in separate packets as shown in Figure \ref{fig:Host transfer data packet}.
For downloads, the host sends the data and the main controller passes the packets to the memory write controller.
In the case of reads, the main controller receives data from the memory read controller, packetizes the data and sends it to the host.
If the amount of data exceeds the 32 data word \ac{mtu} of the \ac{noc}, the data will be fragmented.




\begin{figure}[!t]
  % the [] contains position info e.g. [!t] means here
  \centering
  \captionsetup{justification=centering}

  \begin{minipage}{1\textwidth}
  %\begin{subfigure}{1\textwidth}
    \centering
    \captionsetup{justification=centering, skip=10pt}
    \begin{bytefield}[bitwidth=0.49em, endianness=big]{77}
      \bitheader{0,8,16,24,32,40,48,56,64,68,74,76} \\
      \begin{rightwordgroup}{\scriptsize Header}
        \bitbox{2}{\rotatebox{90}{\tiny SOM}} & \bitbox{7}{\scriptsize src} & \bitbox{2}{\tiny M} & \bitbox{1}{\rotatebox{90}{\tiny prio}} & \bitbox{64}{\tiny destination SSC bitfield} 
      \end{rightwordgroup}  \\
      \begin{rightwordgroup}{\scriptsize Option Cycle}
        \bitbox{2}{\rotatebox{90}{\tiny EOM}} & \bitbox{1}{} & \bitbox{4}{\tiny type} & \bitbox{3}{\tiny pyld\\ \vspace{-0em} type} & \bitbox{1}{} & \bitbox{1}{\tiny P\\ V} & \colorbitbox{optiontype!40}{8}{\tiny config\\ \vspace{-0em}data} & \colorbitbox{optionvalue!40}{3}{\tiny mem\\ dnld} & \colorbitbox{optionvalue!40}{21}{\tiny number of \\ bytes} & \colorbitbox{optiontype!40}{8}{\tiny storage\\ desc} & \colorbitbox{optionvalue!40}{6}{\tiny SSC ID} & \colorbitbox{optionvalue!40}{18}{\tiny storage descriptor\\ pointer}
      \end{rightwordgroup}  \\
      \begin{rightwordgroup}{\scriptsize Data Cycle}
        \bitbox{2}{\rotatebox{90}{\tiny MOM}} & \bitbox{1}{} & \bitbox{4}{\tiny type} & \bitbox{3}{\tiny pyld\\ \vspace{-0em} type} & \bitbox{1}{} & \bitbox{1}{\tiny P\\ V} & \bitbox{32}{\tiny Data} & \bitbox{32}{\tiny Data}
      \end{rightwordgroup}  \\
      \bitbox[]{76}{$\vdots$} \\[1ex]
      \begin{rightwordgroup}{\scriptsize Data Cycle}
        \bitbox{2}{\rotatebox{90}{\tiny EOM}} & \bitbox{1}{} & \bitbox{4}{\tiny type} & \bitbox{3}{\tiny pyld\\ \vspace{-0em} type} & \bitbox{1}{} & \bitbox{1}{\tiny P\\ V} & \bitbox{32}{\tiny Data} & \bitbox{32}{\tiny Data}
      \end{rightwordgroup}  \\
    \end{bytefield}
    \captionsetup{justification=centering, skip=9pt}
    \vspace{-0.5cm}
    \captionof{figure}{Host unsolicited download first \ac{noc} packet}
    \label{fig:Host transfer configuration packet}
  \end{minipage}

  \bigskip
  \vspace{0.5cm}

  \begin{minipage}{1\textwidth}
  %\begin{subfigure}{1\textwidth}
    \centering
    \captionsetup{justification=centering, skip=10pt}
    \begin{bytefield}[bitwidth=0.49em, endianness=big]{77}
      \bitheader{0,8,16,24,32,40,48,56,64,68,74,76} \\
      \begin{rightwordgroup}{\scriptsize Header}
        \bitbox{2}{\rotatebox{90}{\tiny SOM}} & \bitbox{7}{\scriptsize src} & \bitbox{2}{\tiny M} & \bitbox{1}{\rotatebox{90}{\tiny prio}} & \bitbox{64}{\tiny destination SSC bitfield} 
      \end{rightwordgroup}  \\
      \begin{rightwordgroup}{\scriptsize Data Cycle}
        \bitbox{2}{\rotatebox{90}{\tiny MOM}} & \bitbox{1}{} & \bitbox{4}{\tiny type} & \bitbox{3}{\tiny pyld\\ \vspace{-0em} type} & \bitbox{1}{} & \bitbox{1}{\tiny P\\ V} & \bitbox{32}{\tiny Data} & \bitbox{32}{\tiny Data}
      \end{rightwordgroup}  \\
      \bitbox[]{76}{$\vdots$} \\[1ex]
      \begin{rightwordgroup}{\scriptsize Data Cycle}
        \bitbox{2}{\rotatebox{90}{\tiny EOM}} & \bitbox{1}{} & \bitbox{4}{\tiny type} & \bitbox{3}{\tiny pyld\\ \vspace{-0em} type} & \bitbox{1}{} & \bitbox{1}{\tiny P\\ V} & \bitbox{32}{\tiny Data} & \bitbox{32}{\tiny Data}
      \end{rightwordgroup}  \\
    \end{bytefield}
    \captionsetup{justification=centering, skip=9pt}
    \vspace{-0.5cm}
    \captionof{figure}{Host transfer \ac{noc} data only packet}
    \label{fig:Host transfer data packet}
  \end{minipage}
\end{figure}


%The host transfer logic has not been implemented and will be covered in future work.

