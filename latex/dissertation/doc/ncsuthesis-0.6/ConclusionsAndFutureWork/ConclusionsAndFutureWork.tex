
\chapter{Conclusions and Future Work}
\label{sec:Conclusions and Future Work}
\label{sec:chap-seven}

\section[Conclusions]{Conclusions}
\label{sec:Conclusions}

There have been many attempts to accelerate \acp{ann}. Many have shown excellent performance mainly when implementing \acp{cnn}. The improvement mostly comes from the ability to hold kernel weights and/or \ac{an} states in local \ac{sram}. 
Another method of employing local memory is often due to pooling of batch requests, especially in server applications.
This local storage allows the system to take advantage of the low latency and random access benefits of \ac{sram} while performing multiple operations on that data.
When considering applications where this local storage cannot be used effectively, all these implementations suffer a degradation in performance.

This work considers embedded applications where a system is processing requests with a disparate set of \acp{ann}. 
A further assumption is that in embedded applications there are restrictions on power consumption and area.
In this type of target application local \ac{sram} would not be effective and performance is based on \ac{dram} bandwidth.
This work considers \acf{3dic} technology and a customized \acf{3ddram} is proposed. 
The customized \ac{3ddram} combined with a design based on custom instructions and operation descriptors allows the system to achieve high levels of usable memory bandwidth.
There is no doubt existing \ac{cnn} accelerators that take advantage of batch processing achieve a performance that is difficult to better, but applying these systems to this work's target application exposes those systems' \ac{dram} bandwidth limitations.
This work demonstrates a \ac{3dic} system that at the surface provides relatively low \ac{flops}, but considering the target application is memory bound, demonstrates a 3X power improvement and 6X area improvement over similar \ac{ann} systems.
In reality, it is difficult to compare performance against existing \ac{gpu} or \ac{asic} solutions because they target different application spaces. 

\section[Future Work]{Future Work}
\label{sec:Future Work}

This work focused on providing the infrastructure for an expansive architecture.
The design infrastructure implemented was designed to allow additional functionality to be added without a high logic area impact to maintain the validity of the current area study.
This work focused on the \ac{an} state calculations and assumed instructions and configuration tables are pre-loaded.
Therefore future work should include adding the data transfer functionality described in Section \ref{sec:System Overview}.

%The choice of \ac{binary32} made this work more manageable because it allowed
%a certain amount of code reuse.  
There is a level of acceptance that in some cases, lower precision is acceptable in \ac{ann} inference. Therefore further work should include manager and \ac{pe} changes to support \ac{binary16}.
%The choice of \ac{binary32} was in some part convenient and \ac{binary16} might have been a better choice especially as there is a level of acceptance that lower precision is acceptable in \ac{ann} inference.
%Therefore, further work should include manager and \ac{pe} changes to support \ac{binary16}.
In the manager, supporting \ac{binary16} would require additional muxing logic when directing words in the wide \ac{dram} bus to execution lanes as shown in \ref{sec:MRC}, but the bulk of the design should remain relatively intact.
Supporting \ac{binary16} in the \ac{pe} would be relatively straightforward.

This work does not put a high level of importance on the \ac{pe} as the functionality provided by the \ac{pe} is relatively straightforward and primary emphasis was ensuring the \ac{pe} fits within the \ac{3dic} footprint.
However, there is opportunity to research different \ac{pe} architectures such as systolic arrays and a more function-rich \ac{simd}.

This work focused on providing an array of \acp{ssc} to match the array of \acp{dram} interfaces provided by the \ac{diram4}, but further research should include ganging \ac{dram} interfaces to a coarser array of \acp{ssc}.
In practice, this may also be synergistic with alternative \ac{pe} architectures, such as employing a \ac{pe} with a large systolic array \cite{jouppi2017datacenter}.

The instruction decode logic is currently coded using an \ac{fsm}.  In accordance with creating an expansive architecture it might be a better to use a small processor to decode and control system functions.

The \ac{noc} employed in this work was primarily to create a representative \ac{noc} for the area and power analysis.
Therefore, there are opportunities to explore alternative \ac{noc} designs including dynamically changing the \ac{noc} routing table to meet the processing needs of different \ac{ssc} assignments.
This includes considering the routing of the \ac{an} state replication associated with a particular operations.

\iffalse
  \item Half-Precision PE
%    \2 Modifying the PE to support 16-bit FP would be relatively trivial.
  \item Additional PE pipelining
%    \2 Adding layers of processing to the PE would effectively increase the operations per second.
%     \3 The PE area utilization is relatively low so it could support some additional pipelining.
%     \3 Using a reduction array would effectively double the FLOPs.
%     \3 Looking at storing weights and inputs for a pipelined PE would be required and the manager would likely be customized to the PE, but the belief is the Manager design and complexity should be similar.
  \item Systolic PE
%    \2 Systolic arrays have shown efficacy when used in server applications such as TPU, but its not clear how effective they would be in an embedded application.
%      \3 Consider how to effectively feed a systolic array PE.
%      \3 Consider types of systolic arrays.
%      \3 Consider situations such as multiple Managers feeding a single PE. For example, a systolic PE might use two managers to feed X and Y inputs of the array.
\fi

